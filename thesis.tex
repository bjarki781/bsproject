\documentclass[a4paper, 11pt, english]{article}
\usepackage[top=3.17cm, bottom=3.27cm, left=3.5cm, right=3.5cm]{geometry}
%\setlength{\parindent}{2.5em}
\setlength{\parskip}{6pt plus 2pt minus 1pt}
{\renewcommand{\arraystretch}{1.3}

\usepackage[T1]{fontenc}
\usepackage[utf8]{inputenc}
\usepackage{latexsym}
\usepackage{amsmath,amsthm,amssymb,amsfonts}
\usepackage{eucal}
\usepackage[english]{babel}
\usepackage{tikz}
\usepackage{epsfig}
\usepackage{amssymb, graphicx, amsmath, amsthm}
\usepackage{todonotes}
\usepackage[hidelinks]{hyperref}
\usepackage{bm}
\usepackage{appendix}
\usepackage{multicol}
\usepackage[useregional]{datetime2}
\usepackage{listings}
\usepackage{tikz-cd}
\usepackage{caption}

\RequirePackage{graphicx}
\RequirePackage{hyperref}
\newcommand{\parm}{\mathord{\color{black!33}\bullet}}

\newcommand{\breath}{\vspace{6pt plus 2pt minus 1pt}\noindent}

\newcommand{\pattern}[4]{
  \raisebox{0.6ex}{
  \begin{tikzpicture}[scale=0.3, baseline=(current bounding box.center), #1]
    \foreach \x/\y in {#4}
      \fill[NE-lines] (\x,\y) rectangle +(1,1);
    \draw (0.01,0.01) grid (#2+0.99,#2+0.99);
    \foreach \x/\y in {#3}
      \filldraw (\x,\y) circle (5pt);
  \end{tikzpicture}}
}

\newcommand{\patternrule}{ \to \!}

\newtheorem{theorem}{Theorem}[section]
\newtheorem{proposition}[theorem]{Proposition}
\newtheorem{lemma}[theorem]{Lemma}
\newtheorem{corollary}[theorem]{Corollary}
\newtheorem{openproblem}[theorem]{Open Problem}
\newtheorem*{openproblem*}{Open Problem}
\newtheorem{conjecture}[theorem]{Conjecture}

\theoremstyle{definition}
\newtheorem{definition}[theorem]{Definition}
\newtheorem{remark}[theorem]{Remark}
\newtheorem*{remark*}{Remark}
\newtheorem{example}[theorem]{Example}
\newtheorem*{example*}{Example}

\newcommand{\eps}{\epsilon}
\newcommand{\NN}{\mathbb{N}}
\newcommand{\ZZ}{\mathbb{Z}}
\newcommand{\QQ}{\mathbb{Q}}
\newcommand{\etal}{et~al.}
\newcommand{\Sym}{\mathcal{S}}
\newcommand{\Pow}{\mathcal{P}}
\newcommand{\Ocal}{\mathcal{O}}
\newcommand{\emptyword}{\varepsilon}
\newcommand{\id}{\mathrm{id}}
\newcommand{\rid}{\overline{\id}}
\newcommand{\from}{\leftarrow}
\newcommand{\eq}{\leftrightarrow}
\newcommand{\eqstar}{\stackrel{*}{\leftrightarrow}}
\newcommand{\tostar}{\stackrel{*}{\to}}
\newcommand{\toplus}{\stackrel{+}{\to}}
\newcommand{\fromstar}{\stackrel{*}{\from}}
\newcommand{\nf}[1]{{#1\!\downarrow}}
\newcommand{\floor}[1]{\lfloor#1\rfloor}
\newcommand{\bracket}[1]{[\mskip1mu #1\mskip1mu]}
\DeclareMathOperator{\olap}{\mathcal{O}}
\DeclareMathOperator{\dom}{dom}
\DeclareMathOperator{\img}{img}
\DeclareMathOperator{\des}{des}
\DeclareMathOperator{\maj}{maj}
\DeclareMathOperator{\ovl}{ovl}

\begin{document}

\title{Enumerating Hertzsprung equivalence classes defined by partitions of the symmetric group of degree 3}
\author{Bjarki Hartjenstein Gunnarsson \\[0.6cm]{\small Advisor: Anders Claesson}}
\date{\today}
\maketitle

\begin{abstract}
    Hertzsprung patterns are permutation patterns in which both positions and values are required to
    be adjacent. A family of sets of Hertzsprung patterns induces equivalence classes by
    a pattern-replacement equivalence. In this paper, we introduce a method that allows us 
    to automatically enumerate these equivalence classes for some families. We manage to find the
    generating functions which enumerate the equivalence classes defined by a majority of the set partitions of $\Sym_3$.

    \breath \emph{Keywords}: permutation pattern, pattern-rewriting system
\end{abstract}

\section{Introduction}
Permutation patterns are sub-permutations that we can match with parts of other
permutations under various constraints. 
Classical permutation patterns do not have any constrains on their matches but they can
be constrained such that the positions or values of the matches must be adjacent.
Hertzsprung patterns constrain both positions and values to be adjacent. 
These constraints can be formalized by mesh patterns \cite{claesson:2011}.

In this demonstration we make pattern matches bold.  Let us treat $132$ as a classical
pattern, then we would have the following matches in the permutation $15432$:
\[
    \bm{154}32
\]
\[
    \bm{1}54\bm{32}.
\]
If we look again at the same pattern and permutation, but assume that the
pattern match is constrained so that positions must be adjacent, then we have the match
\[
    \bm{154}32.
\]
Again, looking at the same pattern and permutation, but assuming that the
pattern is constrained so that the values must be adjacent we have the following
match:
\[
    \bm{1}54\bm{32}.
\]

Hertzsprung patterns constrain the match so both values and positions must be
adjacent. The example permutation above avoids the Hertzsprung pattern $132$, that is, it does not have a match
within the permutation. But as a showcase, given the pattern $132$ and the permutation $12435$ we
have the Hertzsprung match
\[
    1\bm{243}5.
\]

From this point onward, whenever we mention patterns we are referring to
Hertzsprung patterns.

Finally, a Hertzsprung pattern that matches the beginning of a permutation is
called the (Hertzsprung) \emph{prefix} of that permutations. Similarly, a pattern
that matches the end of a permutation we call a (Hertzsprung) \emph{suffix}.

\subsection{Equivalences, equivalence classes, and motivation}
We can define an equivalence relation using patterns. For a permutation with a certain pattern
match, we can look at the transposition of that pattern match into another as an equivalent
permutation. That is \emph{pattern-replacement equivalence}. Let us look at some examples.

Say we consider $123$ and $321$ to be equivalent. Then given the
permutation with a pattern match $123$ marked in bold,
\[
  5\bm{234}1,
\]
we rearrange the match into the equivalent pattern $321$, and get
\[
  5\bm{432}1.
\]
As a consequence of the patterns $123$ and $321$ being equivalent the permutations $52341$ and
$54321$ land in the same equivalence class. That is, the two permutations are equivalent up to the
pattern-replacement equivalence induced by the equivalence of $123$ and $321$.

We put patterns that can replace each other into sets, and then collect those sets together. 
\[
    \{ \{\alpha, \beta, \gamma, \dots \}, \{\lambda, \mu, \nu, \dots \}, \dots
    \}.
\]
and we have what we call, in this paper, a family of equivalences.

The collection of sets above is a shorthand for
\[
    \{ \{ \alpha, \beta \}, \{ \alpha, \gamma \}, \dots, \{ \lambda, \mu \}, \{
        \lambda, \nu \}, \dots \},
\] 
which we call the canonical form. The shorthand notation is equivalent to the canonical form since
we are working with equivalence relations, and they are transitive.
That is, we can replace the pattern $\beta$ with $\alpha$, and then $\alpha$ with $\gamma$, for
example, to get that $\beta$ and $\gamma$ are in the same equivalence class.

Note that there are sometimes more than one canonical form for a family of sets. For each 
set of equivalent patterns, we can choose one common member. Then we create pairs of that common
element and every other element. In the example above we choose $\alpha$ and $\lambda$ as the common members
but they could be any other member of the respective sets.

In this paper, we denote the set of permutations of length $n$ as $\Sym_n$, also known as the
symmetric group of $n$ letters. We then define the group of all permutations as $\Sym =
\bigcup_{n \geq 1} \Sym_n$.

Again, a collection of sets of equivalent patterns induces a pattern-replacement equivalence.
To put it in more formal language, a family of equivalences $J$ will induce an equivalence relation
which we call $\equiv_J$. We say $u \pi v \equiv_J u \rho v$ if $\{ \pi, \rho \} \in J$, with $u,v
\in \Sym$.

For example, if we say $J = \{ \{ 132, 321 \} \}$, then we consider the patterns $132$ and $321$ to
be equivalent by the relation $\equiv_J$. That will e.g. result in
\[
    1\bm{243}5 \equiv_J 1\bm{432}5
\]
or
\[
  \bm{132}4\bm{765} \equiv_J \bm{321}4\bm{576}.
\]

An equivalence relation on $\Sym = \bigcup_{n \geq 1} \Sym_n$ splits each $\Sym_n$
into a set of equivalence classes which we want to enumerate for each $n$. 

Linton et al. has already made some headway on enumerating equivalence classes of patterns that constrain
positions to be adjacent on one hand, and patterns that constrain values to be adjacent on the other.
Not much work had been done on enumerating equivalence classes of patterns that constrain both
positions and values, Hertzsprung patterns, until the spring of 2021. Then Claesson published a
preprint of his article \cite{claesson:2021} that gives a semi-automatic way of enumerating
equivalence classes of the Hertzsprung type.

To enumerate these different equivalence classes we will introduce Claesson's 
pattern-rewriting systems. They are the only part of the method given by Claesson that requires
some manual work. The work involves finding systems that fulfill a certain property, namely
\emph{termination}.

\subsection{Pattern-rewriting systems}
A \emph{pattern-rewriting system} is a subset $R \subset \bigcup_{n} (\Sym_n \times \Sym_n)$.
We use the notation $\pi \patternrule \rho$ for $(\pi, \rho)$ when referring
to elements of a pattern-rewriting system $R$.

A juxtaposition of permutations signifies a direct sum. If $\pi=123$ and
$\tau=321$, then $\pi\tau=123654$. The left argument is left unchanged while the
digits of the right one are incremented by the length of the left argument.

As for variable naming, we use the variables $a$ and $b$ for whole permutations, 
the variables $\pi$, $\rho$, $\tau$ and $\sigma$ for patterns.
Finally, we use the variables $u$ and $v$ for sub-permutations, permutations that make up larger
permutations.

We can make a pattern-rewriting system induce a \emph{rewrite relation} on $\Sym$ in the following way: We say
$a \to_R b$ for $a, b \in \Sym_k$ if there exists $\pi \patternrule \rho \in R$ so
we can write $a = u \pi v$ and $b = u \rho v$, where
$u$ and $v$ are in $\Sym$. Note that the relation is
length-preserving. For example, if $R = \{123 \patternrule 321\}$, then $1234 \to_R 3214$. 

We denote the domain of $R$ by $\dom(R) = \{\pi : \pi \patternrule
\rho \in R\}$. For the image of $R$ we write $\img(R) = \{\rho : \pi
\patternrule \rho \in R\}$. We write the reflexive transitive closure of
$\to_R$ as $\tostar_R$. 

An element $x$ in $\Sym$ is said to be in \emph{normal form} under the rewrite relation $\to_R$
if there is no $y \in \Sym$ so that $x \to_R y$. A rewrite relation $\to_R$ is said to be
\emph{terminating} if there exist no infinite chains $x_0 \to_R x_1 \to_R \cdots$. If
$\to_R$ is terminating, then every element has at least one normal form. 

We say a rewrite relation is \emph{confluent} if and only if
$$
    y_1 \fromstar x \tostar y_2 \Longrightarrow 
    \exists z. \  y_1 \tostar z \fromstar y_2.
$$
Intuitively, if $\to_R$ is confluent, it implies that every path we can
take using $\to_R$, from a given permutation, will end up in the same place. 

If a system is both terminating and confluent, then every element has one and
only one normal form.

\begin{corollary}
    If $R$ is a terminating and confluent pattern-rewriting system, then the set of
    $\dom(R)$-avoiding permutations in $S_n$, which we also write $S_n(\dom(R))$, is a complete set of
    representatives for $S_n / \equiv_R$, the set of equivalence classes of $\equiv_R$.
\end{corollary}

\subsection{Generating systems from a family of equivalences}

We can generate rewrite systems from families of equivalences. Our criteria for a pattern-rewriting
system $R$ generated from a family $J$ is that the equivalence closure of $R$ is equal to the
equivalence relation generated by $J$. 
Practically, we generate a system from a family of equivalences by taking each pair $\{\alpha, \beta
\}$ in the canonical form, and make into the rule $\alpha \patternrule \beta$ or $\beta \patternrule
\alpha$. The number of different rewrite systems we can generate from $J$ is $2^{|J|}$.

\begin{definition}
  We say a pattern-rewriting system $R$ is \emph{disjoint} if 
  \[
    \dom(R) \cap \img(R) = \emptyset.
  \]
\end{definition}

\begin{example}
  The pattern equivalence $\{ \alpha, \beta \}$ induces an equivalence relation.
  That equivalence relation is equal to the equivalence closure of two different pattern-rewriting
  relations: The relation induced by $\{ \alpha \patternrule \beta \}$ or the relation induced by
  $\{ \beta \patternrule \alpha \}$. Both systems are disjoint.

  The pattern-replacement equivalence induced by $\{ \alpha, \beta, \gamma \}$ is
  equal to the equivalence closure of six different pattern-rewriting
  relations: The relation induced by $\{ \alpha \patternrule \beta, \beta \patternrule \gamma \}$,
  $\{ \alpha \patternrule \beta, \alpha \patternrule \gamma \}$, etc. We see that the first example is
  not disjoint while the second one is. If we only look at disjoint systems, then we get
  $\{ \alpha \patternrule \gamma, \beta \patternrule \gamma \}$,
  $\{ \alpha \patternrule \beta, \gamma \patternrule \beta \}$,
  $\{ \beta \patternrule \alpha, \gamma \patternrule \alpha \}$.
  They all share the property of having a single pattern in their image.
\end{example}

\begin{definition}
  We say a pattern-rewriting system is \emph{regular} if it is composed of rules whose patterns are all
  of the same length.
\end{definition}

\begin{example}
  The system
  \begin{gather*}
    123 \patternrule 321 \\
    213 \patternrule 321
  \end{gather*}
  is regular while the system
  \begin{gather*}
    12 \patternrule 21 \\
    123 \patternrule 321 
  \end{gather*}
  is not.
\end{example}

It is easy to make a terminating system confluent, but then it is rather difficult to find
terminating systems. Later in this paper, we give a lemma that makes it easy to prove that a certain
type of system is terminating. We then show that these kinds of systems can be generated from the
majority of equivalence relations defined by partitions of $\Sym_3$.

\subsection{Increasing statistics}
It is rather difficult to determine if a system is terminating. Claesson introduced a method to
do so involving \emph{increasing statistics}.
A \emph{statistic} is any function $f : \Sym \to \mathbb{N}$. We call $f$ an increasing
statistic with respect to $R$ if $a \to_R b$ implies that $f(a) < f(b)$.

\begin{lemma}
    Let $R$ be a pattern-rewriting system. If there exists an increasing
    statistic with respect to $R$, then $\to_R$ is terminating
    \cite{claesson:2021}.
\end{lemma}

\begin{definition}
    Let the statistic $\Sigma_\tau(a)$ be the sum of the positions of occurrences of
    the pattern $\tau$ in the permutation $a$. We call this statistic the sigma statistic
    parameterized by the pattern $\tau$
    \cite{claesson:2021}.
\end{definition}

Let's take one example from Claesson's paper.
\begin{example}
    Let's look at the equivalence $\{ \{ 12, 21 \} \}$. We can select the system $\{
        21 \to 12 \}$. This relation induced by the system is terminating since the statistic
    $\Sigma_{12}$ is increasing on the relation. It's not
    confluent though, since $321 \to 231$ and $321 \to 312$. So we add the rule
    $231 \patternrule 312$ to the system. The relation is still terminating, and by Lemma
    3.3 in Claesson's paper, we also have a confluent system. We then end up with
    the system 
    \[
      \begin{matrix}
        21 \patternrule 12 \\
        231 \patternrule 312
      \end{matrix}
    \]
    which is the result we are focusing on in this article.

    With Claesson's machinery, and our terminating and confluent system, we find that 
    \[
        \sum_{n \geq 1} \Sym_n(21, 231) = \sum_{m \geq 0} m!(x-x^2)^m
    \]
    which is the generating function for the number of the equivalence classes under
    $\{ \{ 12, 21 \} \}$, as shown by Stanley \cite{stanley:2012}.
\end{example}

We will look more at increasing statistics, and show that if systems fulfill a certain property called
\emph{overlap invariance}, then they are terminating due to a certain increasing statistic.

\section{Finding terminating and confluent systems}
Claesson only found systems corresponding to equivalence relations induced by a
single set of equivalent patterns. We introduce a new type of statistic which can determine the
termination of pattern-rewriting systems generated from families with multiple equivalences.

\subsection{Overlap invariance}
Let us begin by generalizing the sigma statistic Claesson introduced in his paper,
\begin{definition}
    A sigma statistic parameterized by a set of patterns $\Pi$ is defined as
    \[
        \Sigma_\Pi = \sum_{\pi \in \Pi} \Sigma_{\pi},
    \]
\end{definition}
alongside reiterating the definition of $\Ocal$:
\begin{definition}
    For $\sigma, \tau \in T$ define the set of permutations $\Ocal(\sigma,
    \tau)$ by stipulating that $\pi \in \Ocal(\sigma, \tau)$ if and only if
    \begin{enumerate}
    \item $\sigma$ is a proper prefix of $\pi$,
    \item $\tau$ is a proper suffix of $\pi$ and
    \item $|\pi| < |\sigma| + |\tau|$.
    \end{enumerate}
    In this paper, we call the elements of $\Ocal(\sigma, \tau)$ the \emph{clusters} of $\sigma$ with $\tau$.
    If the set $\Ocal(\sigma, \tau)$ is non-empty, we say $\sigma$ \emph{clusters} with $\tau$.
    (Note we use cluster both as a verb and a noun.)

    For example, we have $\Ocal(123, 123) = \left\{ 1234, 12345 \right\}$, $\Ocal(213, 123) = \left\{
    21345 \right\}$, $\Ocal(231, 213) = \emptyset$.
\end{definition}

Then, we can define a cluster \emph{overlap} and the function $\ovl$.
\begin{definition}
  When two patterns or permutations cluster we call the bit they share their \emph{overlap}.
  For permutations $a$ and $b$, we define $\ovl(a,b)$ as the maximum length of their overlaps if they 
  cluster. If they do not cluster, then we set $\ovl(a,b)=0$, the length of their overlap is zero.

  Symbolically, we can say
  \[
    \ovl(\sigma, \tau) = 
      \begin{cases}
        \max\limits_{\kappa \in \Ocal(\sigma, \tau)} |\sigma| + |\tau| - |\kappa| & \mbox{ if } \Ocal(\sigma, \tau) \neq \emptyset \\
        0 & \text{ if $\Ocal(\sigma, \tau) = \emptyset$}.
      \end{cases}
  \]

  For example:
  \begin{align*}
    \ovl(213, 123) = 1 & \text{ since } \Ocal(213, 123) = \{ 21\underline{3}45 \}, \\
    \ovl(123, 123) = 2 & \text{ since } \Ocal(123, 123) = \left\{ 1\underline{23}4, 12\underline{3}45
  \right\}, \\
  \ovl(123, 321) = 0 & \text{ since } \Ocal(123, 321) = \emptyset.
  \end{align*}
\end{definition}

We are able to define \emph{overlap invariance with respect to a set of patterns $T$}.
\begin{definition}
    Let $R$ be a pattern-rewriting system. If
    \[
        \pi_i = \rho_i, |\pi|-m \leq i \leq n \  \textnormal{where}
        \ m = \max\nolimits_{\tau \in T} \ovl(\pi, \tau)
    \]
    for each $\pi \patternrule \rho \in R$,
    then we say the system is \emph{overlap invariant with respect to $T$} or \emph{$T$-overlap
    invariant}.
\end{definition}

The term overlap invariance with respect to $T$ refers to the fact that if a left-hand side of a rule in the system clusters
with some pattern in $T$ to its right, then that rule does not alter the overlap.

\begin{example}
  The system
  \begin{align*}
    123 \patternrule 213 \label{rule1} \tag{1} \\
    312 \patternrule 132 \label{rule2} \tag{2}
  \end{align*}
  is $\img(R)$-overlap invariant. The left-hand side of \eqref{rule1} clusters with the right-hand side of
  \eqref{rule2} and $123_3 = 213_3 = 3$. The left-hand side of \eqref{rule2} clusters with neither 213 nor 132.

  If we apply a rule on a domain pattern that clusters with an image pattern, then we
  do not alter the overlap of the two patterns. Therefore the image pattern to the right stays the
  same, and clusters with a new image pattern.
  
  Let us look at the following permutation, which has the occurrences of $\img(R)$ marked bold and
  occurrences of $\dom(R)$ underlined,
  \[
    \underline{12\bm{3}}\bm{54}.
  \]
  We apply the rule $123 \patternrule 213$ to get the permutation
  \[
    \bm{21\bm{3}}\bm{54}.
  \]
  Since the system is overlap invariant with respect to $\img(R)$ the occurrence of the pattern 132 was left unchanged while
  an occurrence of 213 was added.
\end{example}

This leads us to a new lemma:
\begin{lemma}
  If $R$ is a pattern-rewriting system that is regular, disjoint, and $\img(R)$-overlap invariant,
  then $\Sigma_{\img(R)}$ is an increasing statistic with respect to $R$. 
\end{lemma}
\begin{proof}
  Since the system is disjoint there is no rule $a \patternrule b$ in $R$ with $a, b \in \img(R)$.
  That rule would not move an image pattern to the right, just replace it with another one, which is
  not enough to increase the statistic value of the resulting permutation.

  Also, we can assume that there is no pattern in $\img(R)$ that occurs in a pattern in $\dom(R)$
  because the $R$ is regular.

  Let $a \to_R b$. Then there is a rule $\pi \patternrule \rho$ in
  $R$ so that $u \pi v \to_R u \rho v$, where $a = u \pi v$ and $b =
  u \rho v$. 

  If elements of $\img(R)$ is fully contained in either $u$ or $v$, then we
  can safely ignore them. If an element appears in $u\pi$, at the end of $u$ and the beginning of
  $\pi$, then $u \rho$ will have a higher $\Sigma_{\img(R)}$ value since the position of a pattern in $\img(R)$ will
  effectively move to the right. If an element of $\img(R)$ appears in $\pi v$,
  then by overlap invariance with respect to $\img(R)$ it will also appear in $\rho v$.
  So
  \[
      \Sigma_{\img(R)}(a) < \Sigma_{\img(R)}(b)
  \]

\end{proof}

And since an increasing statistic under a system $R$ implies termination of the rewrite relation
induced by $R$, then we have as a corollary:

\begin{corollary}
    Let $R$ be a pattern-rewriting system that is regular, disjoint and, $\img(R)$-overlap invariant. 
    Then $R$ induces a terminating pattern-rewriting relation.
\end{corollary}

\subsection{Making a terminating system confluent}
Lemma $3.3$ in Claesson's paper gives us a convenient test to tell whether a system is
confluent:
\begin{lemma}[Lemma 3.3 \cite{claesson:2021}]
    Let $R$ be a terminating pattern-rewriting system. Let 
    \[
        \Ocal(R) = \bigcup_{\pi_1, \pi_2} \Ocal(\pi_1, \pi_2),
    \]
    where the union is over all pairs $(\pi_1, \pi_2) \in \dom(R) \times \dom(R)$. If, for all
    $\pi \in \Ocal(R)$,
    \[
        \rho_1 \from \pi \to \rho_2 \Longrightarrow \exists \sigma . \ \rho_1 \fromstar \sigma \tostar
        \rho_2 
    \],
    then $\to_R$ is confluent.
\end{lemma}
Using this test and, a method for adding rules we can devise an algorithm for making a terminating
system $R$ confluent. The method we are using is the
naive one: We find the shortest cluster of patterns in $\Ocal(R)$ that has more than one
normal form under the system. Then we add a rule that maps one normal form into another, making
sure that the result $b$ of applying the rule to a permutation $a$ will result in $\Sigma(a) <
\Sigma(b)$.

Below is pseudocode for the algorithm.

\lstset{basicstyle=\footnotesize\ttfamily, frame=single, breaklines=true,
mathescape=true}
\begin{minipage}{\linewidth}
\begin{lstlisting}[title={Algorithm 1}]
confluentize(sys, f):
    in: a terminating system, sys
        an increasing statistic that proves termination of sys, f
    out: a terminating and confluent system or failure

    if sys is confluent then
        return sys

    s := shortest cluster in O(R) with more than one normal form
    rules := { a -> b | a, b are normal forms of s, f(b) - f(a) > 0 }
    if rules = $\emptyset$ then
        return failure

    rule := rule@(a -> b) from rules chosen by maximum f(b) - f(a)
    return confluentize(sys + rule, f)
\end{lstlisting}
\end{minipage}

We keep on adding rules until the test \verb|sys is confluent| gives us a positive answer
or until there are no more sane rules to add.

Let us try to demonstrate that these added rules which make the system confluent do
not make the system non-terminating.
We call the original terminating system $R$ and, the system with the added rules to make it confluent
$R'$.
 
We have to make sure that an application of a rule to a permutation $x$ must either add occurrences of $\img(R)$, move them
to the right while not ruining occurrences of any of the images present in $x$.

In the following graph, we let a juxtaposition of two bold patterns represent a cluster of theirs. A bold
symbol means that the pattern is unaltered, a proper occurrence of the pattern. A normal weight
symbol represents that the pattern has been possibly spoiled by a rule application, it represents the
remnants of a pattern.

To illustrate this new notation, say we have $R = \left\{ 123 \patternrule 321, 132 \patternrule 231
\right\}$. Let us set $\pi_1 = 123$, $\pi_2 = 132$, $\rho_1 = 321$ and $\rho_2 = 231$.
Then the notation allows us to write $\bm{\pi_1 \pi_2}$ for $12354$ and $\pi_1 \bm{\rho_2} =
12453$. In the latter example what was $\pi_1$ is now $124$ which is not an occurrence of $123$ like
it used to be, therefore not marked bold.

Let us draw up a general scenario with a system $R$, and say 
\[
  \left\{ \pi_1 \patternrule \rho_1, \pi_2 \patternrule \rho_2, 
  \_ \patternrule \rho_a, \_ \patternrule \rho_b \right\} \subseteq R.
\]
Here, "\_" is just a placeholder for some pattern we do not care about. Then we can draw the graph
\[
\begin{tikzcd}  
    & \bm{\rho_a \pi_1 \pi_2 \rho_b} \arrow[dl] \arrow[dr] & \\
    \rho_a \underline{\bm{\rho_1}\pi_2}\bm{\rho_b} & & \bm{\rho_a}\underline{\pi_1 \bm{\rho_2}} \bm{\rho_b}
\end{tikzcd}
\]

The middle part of the top node, $\pi_1\pi_2$, is the general form of a permutation that has more than two normal forms in
a non-confluent system. We can either apply a rule from $R$ to the first or second $\pi$, and get two different
permutations. These are represented by the lower two nodes.

The left-hand side of the rule we add with our algorithm is the underlined part of the lower left
node. Similarly, the right-hand side of the rule is the underlined part of the lower right node.

The two $\rho$ symbols that occur at the beginning and the end of each node represent possible
occurrences of patterns from the image of the system. We include them to show that if they occur
that $a \to_{R'} b$ still implies that $\Sigma_{\img(R)}(a) < \Sigma_{\img(R)}(b)$.

The added rule $\bm{\rho_1} \pi_2 \patternrule \pi_1 \bm{\rho_2}$ is also
overlap invariant with respect to $\img(R)$ since $\pi_2 \patternrule \rho_2$ is $\img(R)$-overlap invariant.

We see that the system $R'$ is terminating with respect to original statistic $\Sigma_{\img(R)}$
since the extra confluence rules move occurrences of $\img(R)$ to the right and are also overlap
invariant with respect to $\img(R)$.

\subsection{Examples}
To illustrate the utility of the methods we have introduced we will look at two
examples

\begin{example}
Let $J=\{\{ 132, 312 \}, \{ 213, 231, 321 \}\}$. This family can generate 6
different disjoint rewrite systems. We test each one for $\img(R)$-overlap invariance 
and one that matches is 
\[
    \begin{matrix}
        123 \patternrule 312 \\
        132 \patternrule 312 \\
        231 \patternrule 321.
    \end{matrix}
\]
We then run our \verb|confluentize| algorithm on this terminating system, and get the system
\[
    \begin{matrix}
        123 \patternrule 312 \\
        132 \patternrule 312 \\
        231 \patternrule 321 \\
        3124 \patternrule 1423 \\
        31254 \patternrule 12534.
    \end{matrix}
\]
This system is both confluent and terminating, which is our end goal.
With that kind of system $R$ we can go on enumerating
permutations that avoid $\dom(R)$ with Claesson's method, and get the correpsonding generating
function 
\[
  \sum_{m \geq 0} m! (-3x^3 + x)^m.
\]
 
\end{example}

\begin{example}

Let $J = \{ \{ 132, 213, 231, 312, 321 \} \}$. There exists a system $R$ which is regular,
disjoint, and $\img(R)$-overlap invariant, hence a terminating pattern-rewriting system
\[
    \begin{matrix}
      132 \patternrule 321 \\
      213 \patternrule 321 \\
      231 \patternrule 321 \\
      312 \patternrule 321
    \end{matrix}
\]
which we can make confluent with our algorithm, and get the system
\[
    \begin{matrix}
      132 \patternrule 321 \\
      213 \patternrule 321 \\
      231 \patternrule 321 \\
      312 \patternrule 321 \\
      3214 \patternrule 1432 \\
      32154 \patternrule 21543.
    \end{matrix}
\]
With Claesson's methods, we find that the cluster generating function of that system is
$\frac{-2x^5-3x^4-4x^3}{x^2+x+1}$ and
again by Theorem 2.1 and Corollary 3.2 in his paper, we have that the generating function for the
number of equivalence classes induced by $J$ is
\[
  \sum_{m \geq 0} m! \left( \frac{-2x^5-3x^4-3x^3+x^2+x}{x^2+x+1} \right)^m.
\]

\end{example}

\section{Overview of results}
We can apply the above-mentioned results by testing all regular and disjoint rewrite systems of a
family of equivalences for overlap invariance with respect to $\img(R)$. If it is $\img(R)$-overlap
invariant,
then it is terminating, and we can make it both terminating and confluent system by adding some rules.

There were three small programs \cite{hartjenstein:2021} written for the project: \verb|sysgen| for
generating the systems from the equivalence relations; \verb|maketex| for taking the domain of these
systems, and calculating the generating functions and looking at if their sequences were found in
the OEIS; and finally, \verb|tester| to enumerate manually equivalence classes induced by each
equivalence relation, up to $n=6$. The programs \verb|sysgen| and \verb|tester| were written in
Haskell and \verb|maketex| was written in the SageMath environment. 

We can look at a set partition as a set of disjoint sets, therefore a set partition of a set of
patterns is a set of disjoint sets of patterns, which is the same as a family of
equivalences.

For example, let us look at one set partition of $\Sym_3$, namely 
\[
  \left\{ \left\{ 123, 132 \right\}, \left\{ 213 \right\}, \left\{ 231, 312 \right\}, \left\{ 321
  \right\} \right\}.
\]
We can ignore the singleton sets, they simply represent that its member is equivalent to itself. We
then get the family of equivalences
\[
  \left\{ \left\{ 123, 132 \right\}, \left\{ 231, 312 \right\} \right\}.
\]

To test the efficacy of the overlap invariance idea we look at every set partition~of 
\[
  \Sym_3 = \left\{ 123, 132, 213, 231, 312, 321 \right\}.
\]
We see that all these set partitions can only generate regular systems, since all their patterns are of
the same length.
Those are 203, of those are 155 families of equivalences from which we can derive disjoint and
$\img(R)$-overlap invariant systems. 

If two families of equivalences induce equivalence relations that have the same generating function
that enumerates equivalences classes defined by the relation,
then we say the families are in the same \emph{Wilf-class}.
Calculating the generating functions of those systems gives us nine different Wilf-classes.

\begin{definition}
    We can give a family of equivalences $J$ a size function which is the same
    as the cardinality of a canonical form of the set of equivalences.

    By this definition, we have for example $J=\{ \{ 123, 321 \}, \{ 312, 231, 213 \} \}$ that
    \begin{align*}
      |J| & = \left| \{ \{ 123, 321 \}, \{ 312, 231, 213 \} \} \right| \\
         & = \left| \{ \{ 123, 321 \}, \{ 312, 231 \}, \{ 312, 213 \} \} \right| \\
         & = 3.
    \end{align*}
\end{definition}

Let us collate all the results from Appendix A into a table to show how the
different generating functions are distributed.

The generating function is always on the form $\sum_{m \geq 0}m!(x+C_J(x))^m$ where $C_J$
is the cluster generating function corresponding to a system derived from $J$.

\begin{center}
\begin{tabular}{c|c|c|c}
    $|J|$ & $C_J(x)$ & count & OEIS seq. \\
    \hline
    0 & $0$ & 1 & A000142 \\
    1 & $-x^3$ & 13 & A212580 \\
    2 & $-2x^3$ & 46 & A212432 \\
    2 & $\frac{-x^3-x^4}{x^2+x+1}$ & 2 & A212581 \\
    3 & $-3x^3$ & 45 & \emph{Not present} \\
    3 & $\frac{-2x^3-2x^4-x^5}{x^2+x+1}$ & 10 & A212433 \\
    4 & $-4x^3$ & 4 & \emph{Not present} \\
    4 & $\frac{-4x^3-3x^4-2x^5}{x^2+x+1}$ & 4 & \emph{Not present}\\
    4 & $\frac{-4x^3-2x^4}{x^2+x+1}$ & 1 & \emph{Not present} \\ 
    \hline
    \hline
    5 & $\frac{-5x^3-3x^4-x^5}{x^2+x+1}$ & 1 & \emph{Not present} \\ 
\end{tabular}
\captionof{table}{Cluster generating functions and their frequency}
\end{center}
If we parameterize each $C_J$ with its corresponding $|J|$ value, then we have
three different types of cluster generating functions.

Those types are 
\begin{gather*}
  C'_J(x) = -|J|x^3, \\
  C''_J(x) = \frac{-|J|x^3-(|J|-1)x^4-(|J|-2)x^5}{x^2+x+1}, \\
  C'''_J(x) = \frac{-|J|x^3-(|J|-2)x^4-(|J|-4)x^5}{x^2+x+1}.
\end{gather*}

For $C'''_J$ we only generated one system (Appendix A: (126)) whereas Claesson generated a system for
$J = \{\{ 123, 132, 213, 231, 312, 321 \}\}$, which we did not manage. Its cluster generating function
is also of the type $C'''_J$. We include it in our table to show that the pattern emerges.

Thus, these partitions of $\Sym_3$ are split into three different types
of Wilf-classes, which we can describe by $C'$, $C''$, and $C'''$; and the size
of the partition.

\section{Problems and questions}
There are a few questions that are left unanswered by this paper.
Are there any other Wilf classes that appear for relations defined by partitions on $\Sym_3$? There are
still approximately 50 relations for which we have not found a corresponding Wilf class. 

Not every relation has a corresponding $\img(R)$-overlap invariant system, but still have terminating and
confluent systems. An example of those would be the relation induced by $\{ \{ 213, 132
\} \}$. Its rewrite systems ${\{ 213 \patternrule 132 \}}$ and ${\{
132 \patternrule 213 \}}$ are not $\img(R)$ overlap invariant. 
But for $\{ 213 \patternrule 132 \}$ sigma parameterized by $21$ is an increasing statistic. We
should be able to find these sigma statistics parameterized by smaller patterns algorithmically.
That should be possible by looking at all Hertzsprung patterns that appear on the left- and right-hand
sides of a rule. Then you can tally if a pattern either appears, disappears, or moves when applying the rule.

Some questions that arise regarding the relation between a family of equivalences $J$ and
the cluster generating function $C_J$.
Would similar formulas to $C', C''$, and $C'''$ appear when we would
generate systems from partitions of $\Sym_4$? 
Or more generally, 
can we derive a way to create a cluster generating function directly from a
given family of equivalences? 

On a separate note, the algorithm we look at in this paper does not manage to make all systems
we find are terminating confluent. Does there exist an algorithm that makes all those systems
confluent, or even all terminating systems?

\begin{openproblem}
    Find a terminating statistic for equivalence relations
    defined by partitions of $\Sym_3$ that do not have a corresponding $\img(R)$-overlap invariant system.
\end{openproblem}

\section*{Acknowledgments}
I would like to thank my friends who helped me with this paper. They are, in no specific order: Höskuldur
Logi Hannesson, Oddur Snorrason, Ingunn Lilja Bergsdóttir, Helgi Sigtryggsson, Breki Pálsson and
Tanguy Heesemans. Finally, I would like to thank my advisor Anders Claesson for introducing me to the
field of combinatorics, and giving me this interesting problem to tackle.

\clearpage

\bibliographystyle{plain}
\bibliography{thesis}

\clearpage

\begin{appendices}

\clearpage

\newgeometry{left=2cm,right=2cm,top=2cm,bottom=2cm}
\section{Resulting systems}
We categorize the families of equivalences by what their joint
distribution is, or namely their Wilf-class.
Each heading includes the generating function, the cluster generating function, and if the enumeration
appears in the OEIS, its OEIS sequence.

\begin{multicols}{2}
\allowdisplaybreaks
\begin{tiny}
$$
\begin{matrix}
\sum_{m \geq 0} m! \left(
x
\right)^m
\ 
0
\\
\left[1, 2, 6, 24, 120, 720, 5040, \text{\texttt{...}}\right]
\ 
\texttt{A000142}
\end{matrix}
$$
\vspace{-1em}
\begin{align}
\textnormal{the identity equivalence}
\end{align}
$$
\begin{matrix}
\sum_{m \geq 0} m! \left(
-x^{3} + x
\right)^m
\ 
-x^{3}
\\
\left[1, 2, 5, 20, 102, 626, 4458, \text{\texttt{...}}\right]
\ 
\texttt{A212580}
\end{matrix}
$$
\vspace{-1em}
\begin{align}
\{\{132, 312\}\}
\quad
&
\phantom{.}
&
\begin{matrix}
312 \mapsto 132
\end{matrix}
\\
\{\{132, 231\}\}
\quad
&
\phantom{.}
&
\begin{matrix}
231 \mapsto 132
\end{matrix}
\\
\{\{231, 321\}\}
\quad
&
\phantom{.}
&
\begin{matrix}
231 \mapsto 321
\end{matrix}
\\
\{\{312, 321\}\}
\quad
&
\phantom{.}
&
\begin{matrix}
312 \mapsto 321
\end{matrix}
\\
\{\{132, 321\}\}
\quad
&
\phantom{.}
&
\begin{matrix}
132 \mapsto 321
\end{matrix}
\\
\{\{213, 321\}\}
\quad
&
\phantom{.}
&
\begin{matrix}
213 \mapsto 321
\end{matrix}
\\
\{\{213, 231\}\}
\quad
&
\phantom{.}
&
\begin{matrix}
231 \mapsto 213
\end{matrix}
\\
\{\{213, 312\}\}
\quad
&
\phantom{.}
&
\begin{matrix}
312 \mapsto 213
\end{matrix}
\\
\{\{123, 213\}\}
\quad
&
\phantom{.}
&
\begin{matrix}
213 \mapsto 123
\end{matrix}
\\
\{\{123, 321\}\}
\quad
&
\phantom{.}
&
\begin{matrix}
321 \mapsto 123\\2341 \mapsto 4123
\end{matrix}
\\
\{\{123, 231\}\}
\quad
&
\phantom{.}
&
\begin{matrix}
231 \mapsto 123
\end{matrix}
\\
\{\{123, 312\}\}
\quad
&
\phantom{.}
&
\begin{matrix}
312 \mapsto 123
\end{matrix}
\\
\{\{123, 132\}\}
\quad
&
\phantom{.}
&
\begin{matrix}
132 \mapsto 123
\end{matrix}
\end{align}
$$
\begin{matrix}
\sum_{m \geq 0} m! \left(
-2 x^{3} + x
\right)^m
\ 
-2 x^{3}
\\
\left[1, 2, 4, 16, 84, 536, 3912, \text{\texttt{...}}\right]
\ 
\texttt{A212432}
\end{matrix}
$$
\vspace{-1em}
\begin{align}
\{\{132, 231, 312\}\}
\quad
&
\phantom{.}
&
\begin{matrix}
231 \mapsto 132\\312 \mapsto 132\\2431 \mapsto 4132\\35412 \mapsto 45132
\end{matrix}
\\
\{\{132, 312\}, \{231, 321\}\}
\quad
&
\phantom{.}
&
\begin{matrix}
132 \mapsto 312\\231 \mapsto 321
\end{matrix}
\\
\{\{132, 231, 321\}\}
\quad
&
\phantom{.}
&
\begin{matrix}
132 \mapsto 321\\231 \mapsto 321
\end{matrix}
\\
\{\{132, 231\}, \{312, 321\}\}
\quad
&
V
&
\begin{matrix}
231 \mapsto 132\\213 \mapsto 123
\end{matrix}
\\
\{\{132, 312, 321\}\}
\quad
&
\phantom{.}
&
\begin{matrix}
132 \mapsto 321\\312 \mapsto 321
\end{matrix}
\\
\{\{132, 312\}, \{213, 321\}\}
\quad
&
\phantom{.}
&
\begin{matrix}
312 \mapsto 132\\321 \mapsto 213\\3241 \mapsto 4213\\43512 \mapsto 54132
\end{matrix}
\\
\{\{132, 231\}, \{213, 321\}\}
\quad
&
\phantom{.}
&
\begin{matrix}
231 \mapsto 132\\321 \mapsto 213\\3241 \mapsto 4213\\35421 \mapsto 45213
\end{matrix}
\\
\{\{213, 231, 321\}\}
\quad
&
\phantom{.}
&
\begin{matrix}
213 \mapsto 321\\231 \mapsto 321
\end{matrix}
\\
\{\{213, 312, 321\}\}
\quad
&
\phantom{.}
&
\begin{matrix}
213 \mapsto 321\\312 \mapsto 321
\end{matrix}
\\
\{\{132, 213, 321\}\}
\quad
&
\phantom{.}
&
\begin{matrix}
132 \mapsto 321\\213 \mapsto 321\\3214 \mapsto 1432\\32154 \mapsto 21543
\end{matrix}
\\
\{\{132, 312\}, \{213, 231\}\}
\quad
&
\phantom{.}
&
\begin{matrix}
312 \mapsto 132\\231 \mapsto 213\\2431 \mapsto 4213\\43512 \mapsto 45132
\end{matrix}
\\
\{\{213, 231\}, \{312, 321\}\}
\quad
&
V
&
\begin{matrix}
312 \mapsto 132\\213 \mapsto 123
\end{matrix}
\\
\{\{132, 321\}, \{213, 231\}\}
\quad
&
\phantom{.}
&
\begin{matrix}
321 \mapsto 132\\231 \mapsto 213\\2431 \mapsto 4132\\43521 \mapsto 45132
\end{matrix}
\\
\{\{213, 231, 312\}\}
\quad
&
\phantom{.}
&
\begin{matrix}
231 \mapsto 213\\312 \mapsto 213\\3241 \mapsto 4213\\43512 \mapsto 45213
\end{matrix}
\\
\{\{132, 213, 231\}\}
\quad
&
\phantom{.}
&
\begin{matrix}
132 \mapsto 231\\213 \mapsto 231\\2314 \mapsto 1342\\23154 \mapsto 21453
\end{matrix}
\\
\{\{132, 231\}, \{213, 312\}\}
\quad
&
\phantom{.}
&
\begin{matrix}
231 \mapsto 132\\312 \mapsto 213\\3241 \mapsto 4132\\35412 \mapsto 45213
\end{matrix}
\\
\{\{213, 312\}, \{231, 321\}\}
\quad
&
\phantom{.}
&
\begin{matrix}
213 \mapsto 312\\231 \mapsto 321
\end{matrix}
\\
\{\{132, 321\}, \{213, 312\}\}
\quad
&
\phantom{.}
&
\begin{matrix}
321 \mapsto 132\\312 \mapsto 213\\2431 \mapsto 4132\\35412 \mapsto 54213
\end{matrix}
\\
\{\{132, 213, 312\}\}
\quad
&
\phantom{.}
&
\begin{matrix}
132 \mapsto 312\\213 \mapsto 312\\3124 \mapsto 1423\\31254 \mapsto 21534
\end{matrix}
\\
\{\{123, 213\}, \{132, 312\}\}
\quad
&
\phantom{.}
&
\begin{matrix}
213 \mapsto 123\\312 \mapsto 132
\end{matrix}
\\
\{\{123, 213\}, \{132, 231\}\}
\quad
&
\phantom{.}
&
\begin{matrix}
213 \mapsto 123\\231 \mapsto 132
\end{matrix}
\\
\{\{123, 213\}, \{231, 321\}\}
\quad
&
\phantom{.}
&
\begin{matrix}
213 \mapsto 123\\231 \mapsto 321
\end{matrix}
\\
\{\{123, 213\}, \{312, 321\}\}
\quad
&
\phantom{.}
&
\begin{matrix}
213 \mapsto 123\\312 \mapsto 321
\end{matrix}
\\
\{\{123, 213\}, \{132, 321\}\}
\quad
&
\phantom{.}
&
\begin{matrix}
213 \mapsto 123\\321 \mapsto 132\\2431 \mapsto 4132
\end{matrix}
\\
\{\{123, 213, 321\}\}
\quad
&
\phantom{.}
&
\begin{matrix}
213 \mapsto 123\\321 \mapsto 123\\2341 \mapsto 4123
\end{matrix}
\\
\{\{123, 213, 231\}\}
\quad
&
\phantom{.}
&
\begin{matrix}
213 \mapsto 123\\231 \mapsto 123
\end{matrix}
\\
\{\{123, 213, 312\}\}
\quad
&
\phantom{.}
&
\begin{matrix}
213 \mapsto 123\\312 \mapsto 123
\end{matrix}
\\
\{\{123, 321\}, \{132, 312\}\}
\quad
&
\phantom{.}
&
\begin{matrix}
321 \mapsto 123\\312 \mapsto 132\\2341 \mapsto 4123\\34512 \mapsto 54132
\end{matrix}
\\
\{\{123, 321\}, \{132, 231\}\}
\quad
&
\phantom{.}
&
\begin{matrix}
321 \mapsto 123\\132 \mapsto 231\\2341 \mapsto 4123
\end{matrix}
\\
\{\{123, 321\}, \{213, 231\}\}
\quad
&
\phantom{.}
&
\begin{matrix}
321 \mapsto 123\\231 \mapsto 213\\2341 \mapsto 4123\\43521 \mapsto 45123
\end{matrix}
\\
\{\{123, 321\}, \{213, 312\}\}
\quad
&
\phantom{.}
&
\begin{matrix}
123 \mapsto 321\\312 \mapsto 213\\3214 \mapsto 1432
\end{matrix}
\\
\{\{123, 231, 321\}\}
\quad
&
\phantom{.}
&
\begin{matrix}
123 \mapsto 321\\231 \mapsto 321\\3214 \mapsto 1432
\end{matrix}
\\
\{\{123, 312, 321\}\}
\quad
&
\phantom{.}
&
\begin{matrix}
123 \mapsto 321\\312 \mapsto 321\\3214 \mapsto 1432
\end{matrix}
\\
\{\{123, 132, 321\}\}
\quad
&
\phantom{.}
&
\begin{matrix}
132 \mapsto 123\\321 \mapsto 123\\2341 \mapsto 4123
\end{matrix}
\\
\{\{123, 231\}, \{132, 312\}\}
\quad
&
\phantom{.}
&
\begin{matrix}
231 \mapsto 123\\312 \mapsto 132\\2431 \mapsto 4123\\34512 \mapsto 45132
\end{matrix}
\\
\{\{123, 231\}, \{312, 321\}\}
\quad
&
V
&
\begin{matrix}
321 \mapsto 132\\213 \mapsto 123\\2431 \mapsto 4132
\end{matrix}
\\
\{\{123, 231\}, \{132, 321\}\}
\quad
&
\phantom{.}
&
\begin{matrix}
231 \mapsto 123\\321 \mapsto 132\\2431 \mapsto 4132\\34521 \mapsto 45132
\end{matrix}
\\
\{\{123, 231\}, \{213, 321\}\}
\quad
&
\phantom{.}
&
\begin{matrix}
231 \mapsto 123\\321 \mapsto 213\\3241 \mapsto 4213\\34521 \mapsto 45213
\end{matrix}
\\
\{\{123, 231\}, \{213, 312\}\}
\quad
&
\phantom{.}
&
\begin{matrix}
231 \mapsto 123\\312 \mapsto 213\\3241 \mapsto 4123\\34512 \mapsto 45213
\end{matrix}
\\
\{\{123, 231, 312\}\}
\quad
&
\phantom{.}
&
\begin{matrix}
231 \mapsto 123\\312 \mapsto 123\\2341 \mapsto 4123\\34512 \mapsto 45123
\end{matrix}
\\
\{\{123, 132, 231\}\}
\quad
&
\phantom{.}
&
\begin{matrix}
132 \mapsto 123\\231 \mapsto 123
\end{matrix}
\\
\{\{123, 312\}, \{132, 231\}\}
\quad
&
\phantom{.}
&
\begin{matrix}
312 \mapsto 123\\231 \mapsto 132\\2341 \mapsto 4132\\35412 \mapsto 45123
\end{matrix}
\\
\{\{123, 312\}, \{231, 321\}\}
\quad
&
\phantom{.}
&
\begin{matrix}
123 \mapsto 312\\231 \mapsto 321\\3124 \mapsto 1423
\end{matrix}
\\
\{\{123, 312\}, \{132, 321\}\}
\quad
&
\phantom{.}
&
\begin{matrix}
312 \mapsto 123\\132 \mapsto 321
\end{matrix}
\\
\{\{123, 312\}, \{213, 321\}\}
\quad
&
\phantom{.}
&
\begin{matrix}
312 \mapsto 123\\321 \mapsto 213\\3241 \mapsto 4213\\43512 \mapsto 54123
\end{matrix}
\\
\{\{123, 312\}, \{213, 231\}\}
\quad
&
\phantom{.}
&
\begin{matrix}
312 \mapsto 123\\231 \mapsto 213\\2341 \mapsto 4213\\43512 \mapsto 45123
\end{matrix}
\\
\{\{123, 132, 312\}\}
\quad
&
\phantom{.}
&
\begin{matrix}
132 \mapsto 123\\312 \mapsto 123
\end{matrix}
\\
\{\{123, 132\}, \{231, 321\}\}
\quad
&
\phantom{.}
&
\begin{matrix}
132 \mapsto 123\\231 \mapsto 321
\end{matrix}
\\
\{\{123, 132\}, \{312, 321\}\}
\quad
&
\phantom{.}
&
\begin{matrix}
132 \mapsto 123\\312 \mapsto 321
\end{matrix}
\\
\{\{123, 132\}, \{213, 321\}\}
\quad
&
V
&
\begin{matrix}
231 \mapsto 321\\123 \mapsto 312\\3124 \mapsto 1423
\end{matrix}
\\
\{\{123, 132\}, \{213, 231\}\}
\quad
&
V
&
\begin{matrix}
231 \mapsto 321\\132 \mapsto 312
\end{matrix}
\\
\{\{123, 132\}, \{213, 312\}\}
\quad
&
V
&
\begin{matrix}
231 \mapsto 321\\213 \mapsto 312
\end{matrix}
\end{align}
$$
\begin{matrix}
\sum_{m \geq 0} m! \left(
\frac{-x^{4} - x^{3} + x^{2} + x}{x^{2} + x + 1}
\right)^m
\ 
\frac{-x^{4} - 2 x^{3}}{x^{2} + x + 1}
\\
\left[1, 2, 4, 17, 89, 556, 4011, \text{\texttt{...}}\right]
\ 
\texttt{A212581}
\end{matrix}
$$
\vspace{-1em}
\begin{align}
\{\{231, 312, 321\}\}
\quad
&
\phantom{.}
&
\begin{matrix}
231 \mapsto 321\\312 \mapsto 321
\end{matrix}
\\
\{\{123, 132, 213\}\}
\quad
&
\phantom{.}
&
\begin{matrix}
132 \mapsto 123\\213 \mapsto 123
\end{matrix}
\end{align}
$$
\begin{matrix}
\sum_{m \geq 0} m! \left(
\frac{-x^{5} - 2 x^{4} - 2 x^{3} + x^{2} + x}{x^{2} + x + 1}
\right)^m
\ 
\frac{-x^{5} - 2 x^{4} - 3 x^{3}}{x^{2} + x + 1}
\\
\left[1, 2, 3, 13, 71, 470, 3497, \text{\texttt{...}}\right]
\ 
\texttt{A212433}
\end{matrix}
$$
\vspace{-1em}
\begin{align}
\{\{132, 231, 312, 321\}\}
\quad
&
\phantom{.}
&
\begin{matrix}
132 \mapsto 321\\231 \mapsto 321\\312 \mapsto 321
\end{matrix}
\\
\{\{213, 231, 312, 321\}\}
\quad
&
\phantom{.}
&
\begin{matrix}
213 \mapsto 321\\231 \mapsto 321\\312 \mapsto 321
\end{matrix}
\\
\{\{123, 213\}, \{231, 312, 321\}\}
\quad
&
\phantom{.}
&
\begin{matrix}
213 \mapsto 123\\231 \mapsto 321\\312 \mapsto 321
\end{matrix}
\\
\{\{123, 132, 213, 321\}\}
\quad
&
\phantom{.}
&
\begin{matrix}
132 \mapsto 123\\213 \mapsto 123\\321 \mapsto 123\\2341 \mapsto 4123
\end{matrix}
\\
\{\{123, 132, 213, 231\}\}
\quad
&
\phantom{.}
&
\begin{matrix}
132 \mapsto 123\\213 \mapsto 123\\231 \mapsto 123
\end{matrix}
\\
\{\{123, 132, 213, 312\}\}
\quad
&
\phantom{.}
&
\begin{matrix}
132 \mapsto 123\\213 \mapsto 123\\312 \mapsto 123
\end{matrix}
\\
\{\{123, 132, 213\}, \{231, 321\}\}
\quad
&
\phantom{.}
&
\begin{matrix}
132 \mapsto 123\\213 \mapsto 123\\231 \mapsto 321
\end{matrix}
\\
\{\{123, 132, 213\}, \{312, 321\}\}
\quad
&
\phantom{.}
&
\begin{matrix}
132 \mapsto 123\\213 \mapsto 123\\312 \mapsto 321
\end{matrix}
\\
\{\{123, 231, 312, 321\}\}
\quad
&
\phantom{.}
&
\begin{matrix}
123 \mapsto 321\\231 \mapsto 321\\312 \mapsto 321\\3214 \mapsto 1432
\end{matrix}
\\
\{\{123, 132\}, \{231, 312, 321\}\}
\quad
&
\phantom{.}
&
\begin{matrix}
132 \mapsto 123\\231 \mapsto 321\\312 \mapsto 321
\end{matrix}
\end{align}
$$
\begin{matrix}
\sum_{m \geq 0} m! \left(
-3 x^{3} + x
\right)^m
\ 
-3 x^{3}
\\
\left[1, 2, 3, 12, 66, 450, 3402, \text{\texttt{...}}\right]
\ 
\texttt{Not found}
\end{matrix}
$$
\vspace{-1em}
\begin{align}
\{\{132, 231, 312\}, \{213, 321\}\}
\quad
&
\phantom{.}
&
\begin{matrix}
231 \mapsto 132\\312 \mapsto 132\\321 \mapsto 213\\2431 \mapsto 4132\\3241 \mapsto 4213\\35412 \mapsto 45132\\45213 \mapsto 54132\\43512 \mapsto 54132\\465213 \mapsto 546132
\end{matrix}
\\
\{\{132, 312\}, \{213, 231, 321\}\}
\quad
&
\phantom{.}
&
\begin{matrix}
132 \mapsto 312\\213 \mapsto 321\\231 \mapsto 321\\3124 \mapsto 1432\\32154 \mapsto 21534
\end{matrix}
\\
\{\{132, 213, 231, 321\}\}
\quad
&
\phantom{.}
&
\begin{matrix}
132 \mapsto 321\\213 \mapsto 321\\231 \mapsto 321\\3214 \mapsto 1432\\32154 \mapsto 21543
\end{matrix}
\\
\{\{132, 231\}, \{213, 312, 321\}\}
\quad
&
H
&
\begin{matrix}
213 \mapsto 312\\123 \mapsto 231\\132 \mapsto 231\\2314 \mapsto 1342\\1423 \mapsto 1342\\31245 \mapsto 21453\\31254 \mapsto 21453\\23154 \mapsto 12453\\312645 \mapsto 312564\\231645 \mapsto 231564
\end{matrix}
\\
\{\{132, 213, 312, 321\}\}
\quad
&
\phantom{.}
&
\begin{matrix}
132 \mapsto 321\\213 \mapsto 321\\312 \mapsto 321\\3214 \mapsto 1432\\32154 \mapsto 21543
\end{matrix}
\\
\{\{132, 312, 321\}, \{213, 231\}\}
\quad
&
H
&
\begin{matrix}
123 \mapsto 312\\132 \mapsto 312\\213 \mapsto 231\\3124 \mapsto 1423\\1342 \mapsto 1423\\31254 \mapsto 12534\\23145 \mapsto 21534\\23154 \mapsto 21534\\312564 \mapsto 312645\\231564 \mapsto 231645
\end{matrix}
\\
\{\{132, 321\}, \{213, 231, 312\}\}
\quad
&
\phantom{.}
&
\begin{matrix}
321 \mapsto 132\\231 \mapsto 213\\312 \mapsto 213\\2431 \mapsto 4132\\3241 \mapsto 4213\\35412 \mapsto 54213\\45132 \mapsto 54213\\43512 \mapsto 45213\\546132 \mapsto 465213
\end{matrix}
\\
\{\{132, 213, 231\}, \{312, 321\}\}
\quad
&
V
&
\begin{matrix}
231 \mapsto 132\\312 \mapsto 132\\213 \mapsto 123\\2431 \mapsto 4132\\35412 \mapsto 45132
\end{matrix}
\\
\{\{132, 231, 321\}, \{213, 312\}\}
\quad
&
V
&
\begin{matrix}
123 \mapsto 231\\132 \mapsto 231\\213 \mapsto 312\\2314 \mapsto 1342\\1423 \mapsto 1342\\23154 \mapsto 12453\\31245 \mapsto 21453\\31254 \mapsto 21453\\231645 \mapsto 231564\\312645 \mapsto 312564
\end{matrix}
\\
\{\{132, 213, 312\}, \{231, 321\}\}
\quad
&
\phantom{.}
&
\begin{matrix}
132 \mapsto 312\\213 \mapsto 312\\231 \mapsto 321\\3124 \mapsto 1423\\31254 \mapsto 21534
\end{matrix}
\\
\{\{123, 213\}, \{132, 231, 312\}\}
\quad
&
\phantom{.}
&
\begin{matrix}
213 \mapsto 123\\231 \mapsto 132\\312 \mapsto 132\\2431 \mapsto 4132\\35412 \mapsto 45132
\end{matrix}
\\
\{\{123, 213\}, \{132, 231, 321\}\}
\quad
&
D_2
&
\begin{matrix}
213 \mapsto 123\\312 \mapsto 132\\321 \mapsto 132\\2431 \mapsto 4132\\35412 \mapsto 54132
\end{matrix}
\\
\{\{123, 213\}, \{132, 312, 321\}\}
\quad
&
\phantom{.}
&
\begin{matrix}
213 \mapsto 123\\312 \mapsto 132\\321 \mapsto 132\\2431 \mapsto 4132\\35412 \mapsto 54132
\end{matrix}
\\
\{\{123, 213, 321\}, \{132, 312\}\}
\quad
&
\phantom{.}
&
\begin{matrix}
213 \mapsto 123\\321 \mapsto 123\\312 \mapsto 132\\2341 \mapsto 4123\\34512 \mapsto 54132
\end{matrix}
\\
\{\{123, 213, 321\}, \{132, 231\}\}
\quad
&
\phantom{.}
&
\begin{matrix}
213 \mapsto 123\\321 \mapsto 123\\231 \mapsto 132\\2341 \mapsto 4123\\35421 \mapsto 45123
\end{matrix}
\\
\{\{123, 213, 231, 321\}\}
\quad
&
V
&
\begin{matrix}
132 \mapsto 123\\312 \mapsto 123\\321 \mapsto 123\\2341 \mapsto 4123\\34512 \mapsto 54123
\end{matrix}
\\
\{\{123, 213, 312, 321\}\}
\quad
&
\phantom{.}
&
\begin{matrix}
213 \mapsto 123\\312 \mapsto 123\\321 \mapsto 123\\2341 \mapsto 4123\\34512 \mapsto 54123
\end{matrix}
\\
\{\{123, 213, 231\}, \{132, 312\}\}
\quad
&
\phantom{.}
&
\begin{matrix}
213 \mapsto 123\\231 \mapsto 123\\312 \mapsto 132\\2431 \mapsto 4123\\34512 \mapsto 45132
\end{matrix}
\\
\{\{123, 213, 231\}, \{312, 321\}\}
\quad
&
V
&
\begin{matrix}
312 \mapsto 132\\321 \mapsto 132\\213 \mapsto 123\\2431 \mapsto 4132\\35412 \mapsto 54132
\end{matrix}
\\
\{\{123, 213, 231\}, \{132, 321\}\}
\quad
&
\phantom{.}
&
\begin{matrix}
213 \mapsto 123\\231 \mapsto 123\\321 \mapsto 132\\2431 \mapsto 4132\\34521 \mapsto 45132
\end{matrix}
\\
\{\{123, 213, 231, 312\}\}
\quad
&
\phantom{.}
&
\begin{matrix}
213 \mapsto 123\\231 \mapsto 123\\312 \mapsto 123\\2341 \mapsto 4123\\34512 \mapsto 45123
\end{matrix}
\\
\{\{123, 213, 312\}, \{132, 231\}\}
\quad
&
D_1
&
\begin{matrix}
123 \mapsto 312\\132 \mapsto 312\\213 \mapsto 231\\3124 \mapsto 1423\\1342 \mapsto 1423\\31254 \mapsto 12534\\23145 \mapsto 21534\\23154 \mapsto 21534\\312564 \mapsto 312645\\231564 \mapsto 231645
\end{matrix}
\\
\{\{123, 213, 312\}, \{231, 321\}\}
\quad
&
D_1
&
\begin{matrix}
123 \mapsto 312\\132 \mapsto 312\\231 \mapsto 321\\3124 \mapsto 1423\\31254 \mapsto 12534
\end{matrix}
\\
\{\{123, 213, 312\}, \{132, 321\}\}
\quad
&
\phantom{.}
&
\begin{matrix}
213 \mapsto 123\\312 \mapsto 123\\321 \mapsto 132\\2431 \mapsto 4132\\35412 \mapsto 54123
\end{matrix}
\\
\{\{123, 321\}, \{132, 231, 312\}\}
\quad
&
\phantom{.}
&
\begin{matrix}
321 \mapsto 123\\231 \mapsto 132\\312 \mapsto 132\\2341 \mapsto 4123\\2431 \mapsto 4132\\34512 \mapsto 54132\\45123 \mapsto 54132\\35412 \mapsto 45132\\465123 \mapsto 456132
\end{matrix}
\\
\{\{123, 321\}, \{132, 312\}, \{213, 231\}\}
\quad
&
\phantom{.}
&
\begin{matrix}
321 \mapsto 123\\312 \mapsto 132\\231 \mapsto 213\\2341 \mapsto 4123\\2431 \mapsto 4213\\34512 \mapsto 54132\\43521 \mapsto 45123\\43512 \mapsto 45132
\end{matrix}
\\
\{\{123, 321\}, \{213, 231, 312\}\}
\quad
&
\phantom{.}
&
\begin{matrix}
321 \mapsto 123\\231 \mapsto 213\\312 \mapsto 213\\2341 \mapsto 4123\\3241 \mapsto 4213\\34512 \mapsto 54213\\45123 \mapsto 54213\\43512 \mapsto 45213\\546123 \mapsto 456213
\end{matrix}
\\
\{\{123, 321\}, \{132, 213, 231\}\}
\quad
&
\phantom{.}
&
\begin{matrix}
123 \mapsto 321\\132 \mapsto 231\\213 \mapsto 231\\3214 \mapsto 1432\\2314 \mapsto 1342\\32154 \mapsto 12453\\21543 \mapsto 12453\\23154 \mapsto 21453\\231654 \mapsto 321564
\end{matrix}
\\
\{\{123, 321\}, \{132, 231\}, \{213, 312\}\}
\quad
&
\phantom{.}
&
\begin{matrix}
321 \mapsto 123\\231 \mapsto 132\\312 \mapsto 213\\2341 \mapsto 4123\\3241 \mapsto 4132\\34512 \mapsto 54213\\35421 \mapsto 45123\\35412 \mapsto 45213
\end{matrix}
\\
\{\{123, 321\}, \{132, 213, 312\}\}
\quad
&
\phantom{.}
&
\begin{matrix}
123 \mapsto 321\\132 \mapsto 312\\213 \mapsto 312\\3214 \mapsto 1432\\3124 \mapsto 1423\\32154 \mapsto 12534\\21543 \mapsto 12534\\31254 \mapsto 21534\\312654 \mapsto 321645
\end{matrix}
\\
\{\{123, 231, 321\}, \{132, 312\}\}
\quad
&
\phantom{.}
&
\begin{matrix}
123 \mapsto 321\\231 \mapsto 321\\132 \mapsto 312\\3214 \mapsto 1432\\32154 \mapsto 12534
\end{matrix}
\\
\{\{123, 231, 321\}, \{213, 312\}\}
\quad
&
\phantom{.}
&
\begin{matrix}
123 \mapsto 321\\231 \mapsto 321\\213 \mapsto 312\\3214 \mapsto 1432\\31245 \mapsto 21543
\end{matrix}
\\
\{\{123, 132, 231, 321\}\}
\quad
&
\phantom{.}
&
\begin{matrix}
123 \mapsto 321\\132 \mapsto 321\\231 \mapsto 321\\3214 \mapsto 1432\\32154 \mapsto 12543
\end{matrix}
\\
\{\{123, 312, 321\}, \{132, 231\}\}
\quad
&
H
&
\begin{matrix}
123 \mapsto 321\\132 \mapsto 321\\213 \mapsto 312\\3214 \mapsto 1432\\1423 \mapsto 1432\\32154 \mapsto 12543\\31245 \mapsto 21543\\31254 \mapsto 21543\\321645 \mapsto 321654\\312645 \mapsto 312654
\end{matrix}
\\
\{\{123, 312, 321\}, \{213, 231\}\}
\quad
&
H
&
\begin{matrix}
123 \mapsto 321\\132 \mapsto 321\\213 \mapsto 231\\3214 \mapsto 1432\\1342 \mapsto 1432\\32154 \mapsto 12543\\23145 \mapsto 21543\\23154 \mapsto 21543\\321564 \mapsto 321654\\231564 \mapsto 231654
\end{matrix}
\\
\{\{123, 132, 312, 321\}\}
\quad
&
\phantom{.}
&
\begin{matrix}
132 \mapsto 123\\312 \mapsto 123\\321 \mapsto 123\\2341 \mapsto 4123\\34512 \mapsto 54123
\end{matrix}
\\
\{\{123, 132, 321\}, \{213, 231\}\}
\quad
&
\phantom{.}
&
\begin{matrix}
123 \mapsto 321\\132 \mapsto 321\\213 \mapsto 231\\3214 \mapsto 1432\\1342 \mapsto 1432\\32154 \mapsto 12543\\23145 \mapsto 21543\\23154 \mapsto 21543\\321564 \mapsto 321654\\231564 \mapsto 231654
\end{matrix}
\\
\{\{123, 132, 321\}, \{213, 312\}\}
\quad
&
\phantom{.}
&
\begin{matrix}
123 \mapsto 321\\132 \mapsto 321\\213 \mapsto 312\\3214 \mapsto 1432\\1423 \mapsto 1432\\32154 \mapsto 12543\\31245 \mapsto 21543\\31254 \mapsto 21543\\321645 \mapsto 321654\\312645 \mapsto 312654
\end{matrix}
\\
\{\{123, 231\}, \{132, 312, 321\}\}
\quad
&
H
&
\begin{matrix}
213 \mapsto 321\\123 \mapsto 312\\132 \mapsto 312\\3124 \mapsto 1423\\1432 \mapsto 1423\\32145 \mapsto 21534\\32154 \mapsto 21534\\31254 \mapsto 12534\\321654 \mapsto 321645\\312654 \mapsto 312645
\end{matrix}
\\
\{\{123, 231\}, \{132, 312\}, \{213, 321\}\}
\quad
&
\phantom{.}
&
\begin{matrix}
231 \mapsto 123\\312 \mapsto 132\\321 \mapsto 213\\2431 \mapsto 4123\\3241 \mapsto 4213\\34512 \mapsto 45132\\34521 \mapsto 45213\\43512 \mapsto 54132
\end{matrix}
\\
\{\{123, 231\}, \{213, 312, 321\}\}
\quad
&
H
&
\begin{matrix}
213 \mapsto 321\\123 \mapsto 231\\132 \mapsto 231\\2314 \mapsto 1342\\1432 \mapsto 1342\\32145 \mapsto 21453\\32154 \mapsto 21453\\23154 \mapsto 12453\\321654 \mapsto 321564\\231654 \mapsto 231564
\end{matrix}
\\
\{\{123, 231\}, \{132, 213, 321\}\}
\quad
&
\phantom{.}
&
\begin{matrix}
123 \mapsto 231\\132 \mapsto 321\\213 \mapsto 321\\2314 \mapsto 1342\\3214 \mapsto 1432\\23154 \mapsto 12543\\21453 \mapsto 12543\\32154 \mapsto 21543\\321564 \mapsto 231654
\end{matrix}
\\
\{\{123, 231\}, \{132, 321\}, \{213, 312\}\}
\quad
&
\phantom{.}
&
\begin{matrix}
231 \mapsto 123\\321 \mapsto 132\\312 \mapsto 213\\2431 \mapsto 4132\\3241 \mapsto 4123\\34521 \mapsto 45132\\34512 \mapsto 45213\\35412 \mapsto 54213
\end{matrix}
\\
\{\{123, 231\}, \{132, 213, 312\}\}
\quad
&
\phantom{.}
&
\begin{matrix}
123 \mapsto 231\\132 \mapsto 312\\213 \mapsto 312\\2314 \mapsto 1342\\3124 \mapsto 1423\\23154 \mapsto 12534\\21453 \mapsto 12534\\31254 \mapsto 21534\\312564 \mapsto 231645
\end{matrix}
\\
\{\{123, 231, 312\}, \{132, 321\}\}
\quad
&
\phantom{.}
&
\begin{matrix}
231 \mapsto 123\\312 \mapsto 123\\321 \mapsto 132\\2341 \mapsto 4123\\2431 \mapsto 4132\\34512 \mapsto 45123\\45132 \mapsto 54123\\35412 \mapsto 54123\\456132 \mapsto 465123
\end{matrix}
\\
\{\{123, 231, 312\}, \{213, 321\}\}
\quad
&
\phantom{.}
&
\begin{matrix}
231 \mapsto 123\\312 \mapsto 123\\321 \mapsto 213\\2341 \mapsto 4123\\3241 \mapsto 4213\\34512 \mapsto 45123\\45213 \mapsto 54123\\43512 \mapsto 54123\\456213 \mapsto 546123
\end{matrix}
\\
\{\{123, 132, 231, 312\}\}
\quad
&
\phantom{.}
&
\begin{matrix}
132 \mapsto 123\\231 \mapsto 123\\312 \mapsto 123\\2341 \mapsto 4123\\34512 \mapsto 45123
\end{matrix}
\\
\{\{123, 132, 231\}, \{312, 321\}\}
\quad
&
D_2
&
\begin{matrix}
123 \mapsto 312\\132 \mapsto 312\\231 \mapsto 321\\3124 \mapsto 1423\\31254 \mapsto 12534
\end{matrix}
\\
\{\{123, 132, 231\}, \{213, 321\}\}
\quad
&
\phantom{.}
&
\begin{matrix}
123 \mapsto 231\\132 \mapsto 231\\213 \mapsto 321\\2314 \mapsto 1342\\1432 \mapsto 1342\\23154 \mapsto 12453\\32145 \mapsto 21453\\32154 \mapsto 21453\\231654 \mapsto 231564\\321654 \mapsto 321564
\end{matrix}
\\
\{\{123, 132, 231\}, \{213, 312\}\}
\quad
&
\phantom{.}
&
\begin{matrix}
123 \mapsto 231\\132 \mapsto 231\\213 \mapsto 312\\2314 \mapsto 1342\\1423 \mapsto 1342\\23154 \mapsto 12453\\31245 \mapsto 21453\\31254 \mapsto 21453\\231645 \mapsto 231564\\312645 \mapsto 312564
\end{matrix}
\\
\{\{123, 312\}, \{132, 231, 321\}\}
\quad
&
\phantom{.}
&
\begin{matrix}
123 \mapsto 312\\132 \mapsto 321\\231 \mapsto 321\\3124 \mapsto 1423\\31254 \mapsto 12543
\end{matrix}
\\
\{\{123, 312\}, \{132, 231\}, \{213, 321\}\}
\quad
&
\phantom{.}
&
\begin{matrix}
312 \mapsto 123\\231 \mapsto 132\\321 \mapsto 213\\2341 \mapsto 4132\\3241 \mapsto 4213\\35412 \mapsto 45123\\35421 \mapsto 45213\\43512 \mapsto 54123
\end{matrix}
\\
\{\{123, 312\}, \{213, 231, 321\}\}
\quad
&
\phantom{.}
&
\begin{matrix}
123 \mapsto 312\\213 \mapsto 321\\231 \mapsto 321\\3124 \mapsto 1423\\32145 \mapsto 21534
\end{matrix}
\\
\{\{123, 312\}, \{132, 213, 321\}\}
\quad
&
\phantom{.}
&
\begin{matrix}
123 \mapsto 312\\132 \mapsto 321\\213 \mapsto 321\\3124 \mapsto 1423\\3214 \mapsto 1432\\31254 \mapsto 12543\\21534 \mapsto 12543\\32154 \mapsto 21543\\321645 \mapsto 312654
\end{matrix}
\\
\{\{123, 312\}, \{132, 321\}, \{213, 231\}\}
\quad
&
\phantom{.}
&
\begin{matrix}
312 \mapsto 123\\321 \mapsto 132\\231 \mapsto 213\\2341 \mapsto 4213\\2431 \mapsto 4132\\35412 \mapsto 54123\\43512 \mapsto 45123\\43521 \mapsto 45132
\end{matrix}
\\
\{\{123, 312\}, \{132, 213, 231\}\}
\quad
&
\phantom{.}
&
\begin{matrix}
123 \mapsto 312\\132 \mapsto 231\\213 \mapsto 231\\3124 \mapsto 1423\\2314 \mapsto 1342\\31254 \mapsto 12453\\21534 \mapsto 12453\\23154 \mapsto 21453\\231645 \mapsto 312564
\end{matrix}
\\
\{\{123, 132, 312\}, \{231, 321\}\}
\quad
&
\phantom{.}
&
\begin{matrix}
123 \mapsto 312\\132 \mapsto 312\\231 \mapsto 321\\3124 \mapsto 1423\\31254 \mapsto 12534
\end{matrix}
\\
\{\{123, 132, 312\}, \{213, 321\}\}
\quad
&
\phantom{.}
&
\begin{matrix}
123 \mapsto 312\\132 \mapsto 312\\213 \mapsto 321\\3124 \mapsto 1423\\1432 \mapsto 1423\\31254 \mapsto 12534\\32145 \mapsto 21534\\32154 \mapsto 21534\\312654 \mapsto 312645\\321654 \mapsto 321645
\end{matrix}
\\
\{\{123, 132, 312\}, \{213, 231\}\}
\quad
&
\phantom{.}
&
\begin{matrix}
123 \mapsto 312\\132 \mapsto 312\\213 \mapsto 231\\3124 \mapsto 1423\\1342 \mapsto 1423\\31254 \mapsto 12534\\23145 \mapsto 21534\\23154 \mapsto 21534\\312564 \mapsto 312645\\231564 \mapsto 231645
\end{matrix}
\\
\{\{123, 132\}, \{213, 231, 321\}\}
\quad
&
V
&
\begin{matrix}
231 \mapsto 321\\123 \mapsto 312\\132 \mapsto 312\\3124 \mapsto 1423\\31254 \mapsto 12534
\end{matrix}
\\
\{\{123, 132\}, \{213, 312, 321\}\}
\quad
&
D_1
&
\begin{matrix}
213 \mapsto 123\\312 \mapsto 132\\321 \mapsto 132\\2431 \mapsto 4132\\35412 \mapsto 54132
\end{matrix}
\\
\{\{123, 132\}, \{213, 231, 312\}\}
\quad
&
V
&
\begin{matrix}
231 \mapsto 321\\132 \mapsto 312\\213 \mapsto 312\\3124 \mapsto 1423\\31254 \mapsto 21534
\end{matrix}
\end{align}
$$
\begin{matrix}
\sum_{m \geq 0} m! \left(
\frac{-2 x^{5} - 3 x^{4} - 3 x^{3} + x^{2} + x}{x^{2} + x + 1}
\right)^m
\ 
\frac{-2 x^{5} - 3 x^{4} - 4 x^{3}}{x^{2} + x + 1}
\\
\left[1, 2, 2, 9, 53, 388, 3019, \text{\texttt{...}}\right]
\ 
\texttt{Not found}
\end{matrix}
$$
\vspace{-1em}
\begin{align}
\{\{132, 213, 231, 312, 321\}\}
\quad
&
\phantom{.}
&
\begin{matrix}
132 \mapsto 321\\213 \mapsto 321\\231 \mapsto 321\\312 \mapsto 321\\3214 \mapsto 1432\\32154 \mapsto 21543
\end{matrix}
\\
\{\{123, 213, 231, 312, 321\}\}
\quad
&
V
&
\begin{matrix}
132 \mapsto 123\\213 \mapsto 123\\312 \mapsto 123\\321 \mapsto 123\\2341 \mapsto 4123\\34512 \mapsto 54123
\end{matrix}
\\
\{\{123, 132, 213, 231, 321\}\}
\quad
&
V
&
\begin{matrix}
123 \mapsto 321\\132 \mapsto 321\\231 \mapsto 321\\312 \mapsto 321\\3214 \mapsto 1432\\32154 \mapsto 12543
\end{matrix}
\\
\{\{123, 132, 213, 312, 321\}\}
\quad
&
\phantom{.}
&
\begin{matrix}
132 \mapsto 123\\213 \mapsto 123\\312 \mapsto 123\\321 \mapsto 123\\2341 \mapsto 4123\\34512 \mapsto 54123
\end{matrix}
\\
\{\{123, 132, 213, 231, 312\}\}
\quad
&
\phantom{.}
&
\begin{matrix}
132 \mapsto 123\\213 \mapsto 123\\231 \mapsto 123\\312 \mapsto 123\\2341 \mapsto 4123\\34512 \mapsto 45123
\end{matrix}
\\
\{\{123, 132, 231, 312, 321\}\}
\quad
&
\phantom{.}
&
\begin{matrix}
123 \mapsto 321\\132 \mapsto 321\\231 \mapsto 321\\312 \mapsto 321\\3214 \mapsto 1432\\32154 \mapsto 12543
\end{matrix}
\end{align}
$$
\begin{matrix}
\sum_{m \geq 0} m! \left(
-4 x^{3} + x
\right)^m
\ 
-4 x^{3}
\\
\left[1, 2, 2, 8, 48, 368, 2928, \text{\texttt{...}}\right]
\ 
\texttt{Not found}
\end{matrix}
$$
\vspace{-1em}
\begin{align}
\{\{123, 213, 231, 321\}, \{132, 312\}\}
\quad
&
D_1
&
\begin{matrix}
123 \mapsto 321\\132 \mapsto 321\\231 \mapsto 321\\213 \mapsto 312\\3214 \mapsto 1432\\1423 \mapsto 1432\\32154 \mapsto 12543\\31245 \mapsto 21543\\31254 \mapsto 21543\\321645 \mapsto 321654\\312645 \mapsto 312654
\end{matrix}
\\
\{\{123, 213, 312, 321\}, \{132, 231\}\}
\quad
&
\phantom{.}
&
\begin{matrix}
213 \mapsto 123\\312 \mapsto 123\\321 \mapsto 123\\231 \mapsto 132\\4132 \mapsto 2341\\2341 \mapsto 4123\\34512 \mapsto 54123\\35412 \mapsto 45123\\35421 \mapsto 45123\\456132 \mapsto 456123\\465132 \mapsto 465123
\end{matrix}
\\
\{\{123, 213, 231, 312\}, \{132, 321\}\}
\quad
&
\phantom{.}
&
\begin{matrix}
213 \mapsto 123\\231 \mapsto 123\\312 \mapsto 123\\321 \mapsto 132\\2341 \mapsto 4123\\2431 \mapsto 4132\\34512 \mapsto 45123\\45132 \mapsto 54123\\35412 \mapsto 54123\\456132 \mapsto 465123
\end{matrix}
\\
\{\{123, 132, 231, 321\}, \{213, 312\}\}
\quad
&
\phantom{.}
&
\begin{matrix}
123 \mapsto 321\\132 \mapsto 321\\231 \mapsto 321\\213 \mapsto 312\\3214 \mapsto 1432\\1423 \mapsto 1432\\32154 \mapsto 12543\\31245 \mapsto 21543\\31254 \mapsto 21543\\321645 \mapsto 321654\\312645 \mapsto 312654
\end{matrix}
\\
\{\{123, 132, 312, 321\}, \{213, 231\}\}
\quad
&
D_1
&
\begin{matrix}
213 \mapsto 123\\312 \mapsto 123\\321 \mapsto 123\\231 \mapsto 132\\4132 \mapsto 2341\\2341 \mapsto 4123\\34512 \mapsto 54123\\35412 \mapsto 45123\\35421 \mapsto 45123\\456132 \mapsto 456123\\465132 \mapsto 465123
\end{matrix}
\\
\{\{123, 231\}, \{132, 213, 312, 321\}\}
\quad
&
V
&
\begin{matrix}
321 \mapsto 132\\213 \mapsto 123\\231 \mapsto 123\\312 \mapsto 123\\2431 \mapsto 4132\\2341 \mapsto 4123\\35412 \mapsto 54123\\45132 \mapsto 54123\\34512 \mapsto 45123\\456132 \mapsto 465123
\end{matrix}
\\
\{\{123, 132, 231, 312\}, \{213, 321\}\}
\quad
&
V
&
\begin{matrix}
132 \mapsto 321\\213 \mapsto 321\\231 \mapsto 321\\123 \mapsto 312\\3214 \mapsto 1432\\3124 \mapsto 1423\\32154 \mapsto 21543\\21534 \mapsto 12543\\31254 \mapsto 12543\\321645 \mapsto 312654
\end{matrix}
\\
\{\{123, 312\}, \{132, 213, 231, 321\}\}
\quad
&
\phantom{.}
&
\begin{matrix}
123 \mapsto 312\\132 \mapsto 321\\213 \mapsto 321\\231 \mapsto 321\\3124 \mapsto 1423\\3214 \mapsto 1432\\31254 \mapsto 12543\\21534 \mapsto 12543\\32154 \mapsto 21543\\321645 \mapsto 312654
\end{matrix}
\end{align}
$$
\begin{matrix}
\sum_{m \geq 0} m! \left(
\frac{-2 x^{4} - 3 x^{3} + x^{2} + x}{x^{2} + x + 1}
\right)^m
\ 
\frac{-2 x^{4} - 4 x^{3}}{x^{2} + x + 1}
\\
\left[1, 2, 2, 10, 58, 408, 3110, \text{\texttt{...}}\right]
\ 
\texttt{Not found}
\end{matrix}
$$
\vspace{-1em}
\begin{align}
\{\{123, 132, 213\}, \{231, 312, 321\}\}
\quad
&
\phantom{.}
&
\begin{matrix}
132 \mapsto 123\\213 \mapsto 123\\231 \mapsto 321\\312 \mapsto 321
\end{matrix}
\end{align}
\end{tiny}

\end{multicols}
\end{appendices}

\end{document}
