\documentclass[a4paper, 11pt]{article}
\usepackage[top=3.17cm, bottom=3.27cm, left=3.5cm, right=3.5cm]{geometry}
\setlength{\parindent}{0pt}
\setlength{\parskip}{6pt plus 2pt minus 1pt}

\usepackage[T1]{fontenc}
\usepackage[utf8]{inputenc}
\usepackage{latexsym}
\usepackage{amsmath,amsthm,amssymb,amsfonts}
\usepackage{eucal}
\usepackage[english]{babel}
\usepackage{tikz}
\usepackage{epsfig}
\usepackage{amssymb, graphicx, amsmath, amsthm}
\usepackage{calrsfs}
\DeclareMathAlphabet{\mathpzc}{OT1}{pzc}{m}{it}
\RequirePackage{graphicx}
\RequirePackage{hyperref}

\newtheorem{theorem}{Theorem}[section]
\newtheorem{proposition}[theorem]{Proposition}
\newtheorem{lemma}[theorem]{Lemma}
\newtheorem{corollary}[theorem]{Corollary}
\newtheorem{openproblem}[theorem]{Open Problem}
\newtheorem*{openproblem*}{Open Problem}
\newtheorem{conjecture}[theorem]{Conjecture}

\theoremstyle{definition}
\newtheorem{definition}[theorem]{Definition}
\newtheorem{remark}[theorem]{Remark}
\newtheorem*{remark*}{Remark}
\newtheorem{example}[theorem]{Example}
\newtheorem*{example*}{Example}

\newcommand{\eps}{\epsilon}
\newcommand{\NN}{\mathbb{N}}
\newcommand{\ZZ}{\mathbb{Z}}
\newcommand{\QQ}{\mathbb{Q}}
\newcommand{\etal}{et~al.}
\newcommand{\Sym}{\mathpzc{S}}
\newcommand{\Pow}{\mathcal{P}}
\newcommand{\Ocal}{\mathcal{O}}
\newcommand{\emptyword}{\varepsilon}
\newcommand{\id}{\mathrm{id}}
\newcommand{\rid}{\overline{\id}}
\newcommand{\from}{\leftarrow}
\newcommand{\eq}{\leftrightarrow}
\newcommand{\eqstar}{\stackrel{*}{\leftrightarrow}}
\newcommand{\tostar}{\stackrel{*}{\to}}
\newcommand{\toplus}{\stackrel{+}{\to}}
\newcommand{\fromstar}{\stackrel{*}{\from}}
\newcommand{\nf}[1]{{#1\!\downarrow}}
\newcommand{\floor}[1]{\lfloor#1\rfloor}
\newcommand{\bracket}[1]{[\mskip1mu #1\mskip1mu]}
\DeclareMathOperator{\olap}{\mathcal{O}}
\DeclareMathOperator{\dom}{dom}
\DeclareMathOperator{\img}{img}
\DeclareMathOperator{\des}{des}
\DeclareMathOperator{\maj}{maj}
\DeclareMathOperator{\len}{len}
\DeclareMathOperator{\shr}{shr}

\begin{document}

\title{Enumerating all Hertzsprung symmetric group 3 equivalence classes}
\author{Bjarki Hartjenstein Gunnarsson}
\date{\today}
\maketitle

\begin{abstract}
    In this paper we introduce a method that allows us to systematically find
    terminating systems, given an equivalence relation. This allows us to
    enumerate the equivalence classes the corresponding relation splits
    elements of $\Sym_n$. This in turn allows us to determine the Wilf-classes
    of some family of relations, in this paper all set partitions of $\Sym_3$.
\end{abstract}

\section{Introduction}
Hertzsprung patterns are permutation patterns in which both positions and values are
required to be adjacent.

Juxtaposition of permutations signifies catenation, so if $\pi=123$ and
$\tau=321$ then $\pi\tau=123654$.

\subsection{Equivalence classes and motivation}
An equivalence relation is a family of sets of permutations.  We consider
permutations in those sets to be equivalent.  An equivalence relation on $\Sym =
\cup_{n \geq 1} \Sym_n$ splits each $\Sym_n$ into a set of equivalence classes.
We want to enumerate those classes.  Work has been done on classical patterns
and (other type linton).  We will look exclusively at equivalence relations of
Hertzsprung patterns.  To do this we will introduce pattern-rewriting systems.

\subsection{Pattern-rewriting systems}
A pattern-rewriting system is a subset $R \subset \cup_n \Sym_n \times \Sym_n$. 
This subset induces a relation on $\Sym$ in the following way: We say
permutations $p, q \in \Sym_k$ relate if there exists $(\alpha,\beta) \in R$ so 
we can write $p = \sigma \alpha \tau$ and $q = \sigma \beta \tau$, where
$\sigma$ and $\tau$ are some permutations.
Note that the relation is length-preserving.
For example, if $R = {123 \to 321}$, then $1234$ relates to $3214$.
The domain of $R$ we denote by $\dom(R) = {\alpha : (\alpha, \beta) \in R}$.
The image of $R$ we in turn denote $\img(R) = {\beta : (\alpha, \beta) \in R}$.
We denote the reflexive transitive closure of $R$ as $\tostar$, that
is, any number of applications of $R$.

An element $x$ in $\Sym$ is said to be in normal form under system $R$ if there
is no $y \in \Sym$ so that $x \to_R y$. The relation induced by the system is
said to be terminating if there exist no infinite chains $x_0 \to x_1 \to
\cdots$. If $\to$ is terminating then every element has a normal form. 

A relation is confluent if and only if
$$
    y_1 \leftarrow x \to y_2 \Longrightarrow \quad 
    \exists. y_1 \to z \leftarrow y_2
$$.

If a system is both terminating and confluent, then every element has a one and
only one normal form. Church-Rosser?

\begin{corollary}
    If $R$ is a terminating and confluent pattern-rewriting system, then the set
    of $\dom(R)$-avoiding permutations in $S_n$ is a complete set of
    representives for $S_n / \equiv$, the set of equivalence classes of
    $\equiv$.
\end{corollary}

So the problem is reduced to finding a terminating and confluent system $R$ for
an equivalence relation and then enumerating $\dom(R)$-avoiding permutations.
Section 2 of Anders' paper provides us with a way to do that. \cite{claesson}

With this paper we set out to develop a mechanical test for termination.

When finding systems, we begin with naive systems that terminate, and which we
then make confluent. Naive systems are made up of small systems/rule groups that
correspond to each equivalence in an equivalence relation.

\begin{lemma}
    With a naive system that is terminating, then it's possible to preserve its
    termination when making it confluent.
\end{lemma}
\begin{proof}
    By ensuring that all rules that are added have a positive sigma difference
    we can be certain that a system is still confluent. 
\end{proof}

\section{Finding terminating naive systems}
Claesson only found systems corresponding to equivalence relations containing a single equivalence but we will
introduce a method for finding systems having any number of equivalences.
Each equivalece in an equivalence relation has at least two small systems.
\begin{example}
    The equivalence $123 \equiv 321$ has two rule groups, namely $\{123 \to
    321\}$ and $\{321 \to 123\}$. The same goes for $132 \equiv 312$, $\{132 \to
    312\}$ and $\{312 \to 132\}$.

    Then the relation $\{123 \equiv 321, 132 \equiv 312\}$ has for possible
    naive systems: $\{123 \to 321, 132 \to 312\}$, $\{123 \to 321, 312 \to
    132\}$, etc.
\end{example}
A system is composed of rule classes catenated together.

A statistic is any function $f : S \to \mathbb{N}$. We call $f$ an increasing
statistic if $\pi \to \sigma$ implies that $f(\pi) < f(\sigma)$.

\begin{lemma}
    Let $R$ be a pattern-rewriting system. If there exists an increasing
    statistic with respect to $R$, then $\to$ is terminating.    
\end{lemma}

So the problem becomes finding a terminating and confluent system of a given
equivalence relation. In this paper I will introduce a systematic way to find
terminating systems and explain its backfalls.

First we can reduce the amount we check by the symmetries of the square.
If we show that a system is both terminating and confluent, then all systems
which we can arrive at through mirroring the permutations etc are also
terminating and confluent.

Let's begin focussing on \emph{small systems}. That is, systems, which have only
a single pattern as their image. A small system corresponds to an equivalence.
We then catenate a family the small systems to create a system that correponds
to a respective family of equivalences.

\begin{lemma}
    Let $E$ be a small system. If $\pi \to \rho$ is in $E$ and 
    \begin{equation}
        \sum\nolimits_\rho(u\pi v) < \sum\nolimits_\rho(u \rho v)
    \end{equation}
    then $\sum$ parameterized by the image of $E$ is an increasing statistic on
    $E$.
\end{lemma}
\begin{proof}
    If $\pi \to \rho$ is in $E$ then we see $\sum_\rho(\pi) <
    \sum_\rho(\rho)$.  We see that each application of a rule of the system
    will only make the resulting permutation have a higher sigma value.
\end{proof}

\begin{lemma}
    Let $E$ be a small system with $\{\rho\} = \img(E)$. If
    \[
        \pi_i(-l_i, 0) = \rho(-l_i, 0) \quad \textnormal{where} \quad l_i :=
        \len(\shr(\pi_i, \rho)) \tag{$\ast$} \label{joininvariance}
    \]
    for each $\pi_i \to \rho \in E$ then $E$ fulfills condition
    \eqref{joininvariance}. Note that
    when $l_i = 0$ then $\Ocal(\pi_i, \rho) = \emptyset$.
\end{lemma}
\begin{proof}
    Let us assume that we have a rule in $R$, namely $\pi \to \rho$ and so
    $0 = \sum_\rho(\pi) < \sum_\rho(\rho) = 1$. Then $u \pi v \to u \rho v$. 
    If $\rho$ is fully contained in either $u$ or $v$ then $\to$ won't have any
    implication for those occurences. If $\rho$ appears in $u\pi$ then
    application of the rule will move it further to the right, increasing the
    value of sigma. If $\rho$ appears in $\pi v$ then we assume it's static for
    that part so we will have two occurences of $\rho$, one where $\pi$ was and
    then we did not alter other one.

    We can see that every application of rules from $R$ will increase.
\end{proof}

\begin{lemma}
    If a family of small systems have increasing statistics then the sum of the
    statistics is increasing for the catenation of the family of smalll systems. 
\end{lemma}
\begin{proof}
    Each rule application will increase the sum statistic.
\end{proof}

\begin{definition}
    Let the statistic $\sum_\tau$ be the sum of the positions of occurences of
    the pattern $\tau$. We can use this as a terminating statistic.
    \cite{claesson}.

    A sigma statistic parameterized by a set of patterns is simply defined as
    $$
        \sum\nolimits_{\{\tau_0, ..., \tau_n\}}(x) = \sum_{i=0}^{n} \sum\nolimits_{\tau_i}(x).
    $$.
\end{definition}

With this new definiton we can work with equivalence relations that contain more than one
equivalence.

\begin{theorem}
    Let $R$ be a system composed of the small systems $E_i$ which fulfill
    condition (*). We say ${\rho_i} = \img(E_i)$. Then $\sum\nolimits_{\rho_i}$ 
    is an increasing statistic for each $E_i$ and then by lemma X we have an increasing statistic for $R$.
    The system $R$ is hence terminating.
\end{theorem}

This theorem works very well for finding terminating systems to begin
confluentize, but it does not work for every possible equivalence relation.
Take ... for example
But here sigma parameterized by a smaller statistic, namely $12$, is an
increasing statistic.

So for each equivalence relation we can check if there is an image that fulfills
our lemma and if so, use it.

This works for all systems except those containing the relation $231 \equiv $

\section{Application, result and verification}
There are 10 different combinatorial classes.
\subsection{Programming}
There were three small programs written for the project: mp for generating the
systems from the equivalence relations, mp2 for taking the domain of these
systems and calculating the generating functions and finding if their sequences
were found in the OEIS, and finally tester which calculates the first 5 values
of the sequences manually. mp and tester were written in Haskell and mp2 in SageMath. 


\section{Problems, questions, remarks}
Are there larger systems, for example from $\Sym_4$, where we cannot apply the
methods used in this paper?

\bibliographystyle{plain}
\bibliography{mp}

\end{document}
