\documentclass[openany, a4paper, 11pt, english]{article}
\usepackage[top=3.17cm, bottom=3.27cm, left=3.5cm, right=3.5cm]{geometry}
%\setlength{\parindent}{2.5em}
\setlength{\parskip}{6pt plus 2pt minus 1pt}
{\renewcommand{\arraystretch}{1.3}

\usepackage[T1]{fontenc}
\usepackage[utf8]{inputenc}
\usepackage{latexsym}
\usepackage{amsmath,amsthm,amssymb,amsfonts}
\usepackage{eucal}
\usepackage[english]{babel}
\usepackage{tikz}
\usepackage{epsfig}
\usepackage{amssymb, graphicx, amsmath, amsthm}
\usepackage{calrsfs}
\usepackage{todonotes}
\usepackage[hidelinks]{hyperref}
\usepackage{bm}
\usepackage{appendix}
\usepackage{multicol}
\usepackage[useregional]{datetime2}
\usepackage{listings}
%\usepackage{courier}

\RequirePackage{graphicx}
\RequirePackage{hyperref}
\newcommand{\parm}{\mathord{\color{black!33}\bullet}}

\newcommand{\breath}{\vspace{6pt plus 2pt minus 1pt}\noindent}

\newcommand{\patternrule}{ \mapsto \!}

\newtheorem{theorem}{Theorem}[section]
\newtheorem{proposition}[theorem]{Proposition}
\newtheorem{lemma}[theorem]{Lemma}
\newtheorem{corollary}[theorem]{Corollary}
\newtheorem{openproblem}[theorem]{Open Problem}
\newtheorem*{openproblem*}{Open Problem}
\newtheorem{conjecture}[theorem]{Conjecture}

\theoremstyle{definition}
\newtheorem{definition}[theorem]{Definition}
\newtheorem{remark}[theorem]{Remark}
\newtheorem*{remark*}{Remark}
\newtheorem{example}[theorem]{Example}
\newtheorem*{example*}{Example}

\newcommand{\eps}{\epsilon}
\newcommand{\NN}{\mathbb{N}}
\newcommand{\ZZ}{\mathbb{Z}}
\newcommand{\QQ}{\mathbb{Q}}
\newcommand{\etal}{et~al.}
\newcommand{\Sym}{S}
\newcommand{\Pow}{\mathcal{P}}
\newcommand{\Ocal}{\mathcal{O}}
\newcommand{\emptyword}{\varepsilon}
\newcommand{\id}{\mathrm{id}}
\newcommand{\rid}{\overline{\id}}
\newcommand{\from}{\leftarrow}
\newcommand{\eq}{\leftrightarrow}
\newcommand{\eqstar}{\stackrel{*}{\leftrightarrow}}
\newcommand{\tostar}{\stackrel{*}{\to}}
\newcommand{\toplus}{\stackrel{+}{\to}}
\newcommand{\fromstar}{\stackrel{*}{\from}}
\newcommand{\nf}[1]{{#1\!\downarrow}}
\newcommand{\floor}[1]{\lfloor#1\rfloor}
\newcommand{\bracket}[1]{[\mskip1mu #1\mskip1mu]}
\DeclareMathOperator{\olap}{\mathcal{O}}
\DeclareMathOperator{\dom}{dom}
\DeclareMathOperator{\img}{img}
\DeclareMathOperator{\des}{des}
\DeclareMathOperator{\maj}{maj}
\DeclareMathOperator{\len}{len}
\DeclareMathOperator{\shr}{shr}

\begin{document}

\title{Enumerating Hertzsprung equivalence classes defined by partitions of the symmetric group of degree 3}
\author{Bjarki Hartjenstein Gunnarsson}
\date{\today}
\maketitle

\begin{abstract}
    Hertzsprung patterns are permutation patterns in which both positions and
    values are required to be adjacent. Families of equivalences of these
    patterns induce equivalence classes. In this paper we introduce a method
    that allows us, given an equivalence relation, to enumerate these
    equivalence classes. This in turn allows us to determine the Wilf-classes of
    all the set partitions of $\Sym_3$.
    \breath \emph{Keywords}: permutation pattern, pattern-rewriting system
\end{abstract}

\section{Introduction}
Permutation patterns are powerful tools which can be utilized to solve various
problems. example, example and example. They can be made to obey various
constrains, which are formalized by mesh patterns. \cite{claesson:2011}

Classical permutation patterns do not have any constraints on their matches but then we can
constrain them such that the matches' positions or values must be adjacent.
Hertzsprung patterns constrain both. 

In this demonstration we make pattern matches bold.  Let us treat $132$ as classical
pattern, then we would have the following matches in the permutation $15432$.
\[
    \bm{154}32
\]
\[
    \bm{1}54\bm{32}
\]
We will look again at the same pattern and permutation, but assume that the
pattern match is constrained by positions.
\[
    \bm{154}32
\]
Again, looking at at the same pattern and permutation, but assume that the
pattern is constrained so that the values must be adjacent we have the following
match:
\[
    \bm{1}54\bm{32}.
\]

Hertzsprung patterns constrain the match so both values and positions must be
adjacent. There is no match for the Hertzsprung pattern $132$ in the example
above but as a showcase, given the pattern $132$ and the permutation $1243$ we
have the Hertzsprung match
\[
    1\bm{243}.
\]

\subsection{Equivalence, equivalence classes and motivation}
We can look at patterns as being equivalent. Transposition of each pattern into
the other in a permutation will result in an equivalent permutation. Throughout
this article we represent some equivalent patterns with be a family of sets, signifying
a family of equivalences:
\[
    \{ \{\alpha, \beta, \gamma, \dots \}, \{\lambda, \mu, \nu, \dots \}, \dots
    \}.
\]
This is actually a shorthand for
\[
    \{ \{ \alpha, \beta \}, \{ \alpha, \gamma \}, \dots, \{ \lambda, \mu \}, \{
        \lambda, \nu \}, \dots \}
\]
which we call the canonical form.

A family of equivalences $J$ will induce an equivalence relation which we call $\sim_J$. We
say $u \pi v \sim_J u \rho v$ if $\{ \pi, \tau \} \in J$.

If we say $J = \{ \{ 132, 321 \} \}$ then we consider the patterns $132$ and $321$ to
be equivalent by relation $\sim_J$. That will, for example, result in
\[
    1\bm{243} \sim_J 1\bm{432}
\]

An equivalence relation on $\Sym = \cup_{n \geq 1} \Sym_n$ splits each $\Sym_n$
into a set of equivalence classes which we want to enumerate. 

Linton had already made some headway on enumerating equivalence classes the
first two types but the spring of 2021 Claesson published a preprint of his
article that gives a semi-automatic way of enumerating equivalence classes of
the last type.

In this article we will attempt to give an automatic method for enumerating the
equivalence classes defined by partitions of $\Sym_3$ and some thoughts on the
resulting generating functions.

To enumerate these different equivalence classes this we will introduce
pattern-rewriting systems.

\subsection{Pattern-rewriting systems}
A pattern-rewriting system is a subset $R \subset \cup_{n} \Sym_n \times \Sym_n$.
We use the notation $\alpha \patternrule \beta$ for $(\alpha, \beta)$ when referring
to elements of a pattern-rewriting system $R$. 

Juxtaposition of permutations signifies concatenation, so if $\pi=123$ and
$\tau=321$ then $\pi\tau=123654$ with the left argument being unchanged and the
right pushed up.

We use the variables $\alpha$ and $\beta$ for whole permutations, 
the variables $\pi$, $\rho$, $\tau$ and $\sigma$ for patterns.
Finally we use the variables $u$ and $v$ for slices of permutations.

We can make this subset induce a \emph{rewrite relation} on $\Sym$ in the following way: We say
$p \to_R q$ for $p, q \in \Sym_k$ if there exists $\alpha \patternrule \beta \in R$ so
we can write $p = u \alpha v$ and $q = u \beta v$, where
$u$ and $v$ are some permutations.  Note that the relation is
length-preserving. For example, if $R = \{123 \patternrule 321\}$, then $1234 \to_R 3214$. 

\breath We denote the domain of $R$ by $\dom(R) = \{\alpha : \alpha \patternrule
\beta \in R\}$. The image of $R$ we in turn write $\img(R) = \{\beta : \alpha
\patternrule \beta \in R\}$. We write the reflexive transitive closure of
$\to_R$ as $\tostar_R$. 

An element $x$ in $\Sym$ is said to be in normal form under the relation $\to_R$
if there is no $y \in \Sym$ so that $x \to_R y$. A relation $\to_R$ is said to be
terminating if there exist no infinite chains $x_0 \to_R x_1 \to_R \cdots$. If
$\to_R$ is terminating then every element has at least one normal form. 

We say a relation is confluent if and only if
$$
    y_1 \fromstar x \tostar y_2 \Longrightarrow 
    \exists z. \  y_1 \tostar z \fromstar y_2.
$$
Intuitively, if $\to_R$ is confluent then that implies that every path we can
take using $\to_R$, from a given permutation, will end up in the same place. 

If a system is both terminating and confluent, then every element has a one and
only one normal form.

\begin{corollary}
    If $R$ is a terminating and confluent pattern-rewriting system, then the set
    of $\dom(R)$-avoiding permutations in $S_n$ is a complete set of
    representatives for $S_n / \sim_R$, the set of equivalence classes of
    $\sim_R$.
\end{corollary}

We can derive rewrite relations from a given equivalence relation $\sim_J$.
The only condition we set on those derived rewrite relations is that their
equivalence closure must be equal to the given equivalence relation.
A set of equivalent pairs of patterns $J$ is equal to the set of rewrites to
either direction. 
\begin{example}
    The equivalence $\{ \{123, 321 \} \}$ induces an equivalence relation which is
    equal to the equivalence closure of the rewrite relation induced by two
    different single-image systems, namely $\{123 \patternrule 321\}$ or $\{321
    \patternrule 123\}$. The same goes for $\{ \{ 132, 312 \} \}$, $\{132 \patternrule
    312\}$ or $\{312 \patternrule 132\}$.

    The equivalence closure of the rewrite relation induced by each of these
    four rewrite systems: $\{123 \patternrule 321, 132 \patternrule 312\}$, $\{123
    \patternrule 321, 312 \patternrule 132\}$, $\{321 \patternrule 123, 132
    \patternrule 312\}$, $\{321 \patternrule 123, 312 \patternrule 132\}$ is
    equal to the equivalence relation induced by $\{ \{ 123, 321 \} \}$.

    So for all of these $R$ we have that $\sim_J$ is the same as $\sim_R$.
\end{example}

An element of an canonical family equivalence relation $\{ \alpha, \beta \}$ is
actually equal to the pair $(\alpha, \beta)$ and $(\beta, \alpha)$.
Deriving an pattern-rewriting system from a family of equivalences is in
practice choosing either element of the pair for each element of the canonical
equivalence relation.

So the problem is reduced to generating a terminating and confluent system $R$
given a family of equivalences. We then enumerate $\dom(R)$-avoiding
permutations which is the same thing as enumerating different equivalence
classes.  Section 2 of Claesson's paper provides us with a way to do that.
\cite{claesson:2021}

Our method boils down to finding systems which terminate, and we then try to
make them confluent. These terminating systems are composed of single-image
systems that each are created from a respective equivalence relation.

\subsection{Increasing statistics}
A statistic is any function $f : \Sym \to \mathbb{N}$. We call $f$ an increasing
statistic with respect to $R$ if $\alpha \to_R \beta$ implies that $f(\alpha) < f(\beta)$.

\begin{lemma}
    Let $R$ be a pattern-rewriting system. If there exists an increasing
    statistic with respect to $R$, then $\to_R$ is terminating.    
\end{lemma}

\begin{definition}
    Let the statistic $\sum_\tau(\alpha)$ be the sum of the positions of occurrences of
    the pattern $\tau$ in $\alpha$. We call this statistic the sigma statistic
    parameterized by the pattern $\tau$.
    \cite{claesson:2021}.
\end{definition}

Let's take one example from Claesson's paper.
\begin{example}
    Let's look at the equivalence $\{ \{ 12, 21 \} \}$. We can select the system $\{
        21 \to 12 \}$. This relation induced by the system is terminating since the statistic
    $\sum_{12}$ is increasing on the relation. It's not
    confluent though, since $321 \to 231$ and $321 \to 312$. So we add the rule
    $231 \patternrule 312$ to the system. The relation is still terminating and by Lemma
    3.3 in Claesson's paper we also have a confluent system. We then end up with
    the system 
    \[
        \{ 21 \patternrule 12, 231 \patternrule 312 \}
    \]

    \breath

    By his machinery we find that 
    \[
        \sum_{n \geq 1} \Sym_n(21, 231) = \sum_{m \geq 0} m!(x-x^2)^m
    \]
    which is the joint distribution of the equivalence classes defined by 
    $\{ \{ 12, 21 \} \}$, as found by Stanley CITE.
\end{example}

\section{Finding systems automatically}
Claesson only found systems corresponding to equivalence relations induced by a
single set of equivalent patterns. We will introduce a method for finding
systems for equivalence relations induced by any number of sets.

\subsection{Single-image systems}

Let's begin focusing on \emph{single-image systems}. That is, systems, which
have only a single pattern as their image. Claesson looked at these kinds
of systems. A single-image system corresponds to a set of patterns.
We then join a family the single-image systems to
create a system that corresponds to a respective family of equivalences.

Let us introduce a method for determining a terminating system given an
equivalence of patterns.

\begin{lemma}
    Let $E$ be a single-image system. If $\pi \patternrule \rho$ is in $E$ and 
    \[
        \sum\nolimits_\rho(u\pi v) < \sum\nolimits_\rho(u \rho v) \tag{$\ast$}
        \label{increasingimage}
    \]
    then $\sum$ parameterized by the image of $E$ is an increasing statistic
    with respect to $E$.
\end{lemma}
\begin{proof}
    If $\pi \to \rho$ is in $E$ then we see $\sum_\rho(\pi) <
    \sum_\rho(\rho)$.  We see that each application of a rule of the system
    will only make the resulting permutation have a higher sigma value.
\end{proof}

 We also let $\shr(\alpha,
\beta)$ signify the part where, if $\alpha$ and $\beta$ form a cluster, they
share. $\shr(132, 213) = 32$. We assume that they just have one cluster, since
we're not clustering the same pattern, 123 and 123, for example. We index a
permutation $\alpha$ with $(\alpha)_i$, $1 \leq i \leq |\alpha|$.

\begin{lemma}
    Let $E$ be a single-image system with $\{\rho\} = \img(E)$ and $n = |\rho|$. If
    \[
        (\pi_i)_j = (\rho)_j \quad \textnormal{where} \quad n-\len(\shr(\pi_i,
        \rho)) \leq j \leq n \tag{$\ast\ast$} \label{joininvariance}
    \]
    for each $\pi_i \patternrule \rho \in E$ then $E$ fulfills condition
    \eqref{increasingimage}. \end{lemma}
\begin{proof}
    Let us assume that we have a rule in $E$, namely $\pi \patternrule \rho$ and so
    $0 = \sum_\rho(\pi) < \sum_\rho(\rho) = 1$. Then $u \pi v \to u \rho v$. 
    If $\rho$ is fully contained in either $u$ or $v$ then $\to$ won't have any
    implication for those occurrences. If $\rho$ appears in $u\pi$ then
    application of the rule will move it further to the right, increasing the
    value of sigma. If $\rho$ appears in $\pi v$ then we assume it's static for
    that part so we will have two occurrences of $\rho$, one where $\pi$ was and
    then we did not alter other one.
    We can see that every application of rules from $R$ will increase.
\end{proof}

So by looking at the two lemmas together we have that if a single-image system
fulfills \eqref{joininvariance} then sigma parameterized by the image of $E$ is
an increasing statistic. Hence $E$ is terminating.

Let us turn to families of equivalences which can generate systems with more
than one image.

\subsection{Multi-image systems}
\begin{lemma}
    A family of single-image systems joined together induce an equivalence
    relation that is the same as defined by the family of equivalences
    corresponding to each system.
\end{lemma}
\begin{proof}
    Each $x$ in $\Sym$ has a normal form by Corollary 1.1 and we are still counting
    them, just as with single-image systems.
\end{proof}

\begin{definition}
    A sigma statistic parameterized by a set of patterns $\Pi$ is defined as
    $$
        \sum\nolimits_\Pi = \sum_{\pi \in \Pi} \sum\nolimits_{\pi}.
    $$
    We see this is an increasing statistic for a union of systems if
    each system has an increasing $\sum_{\tau_i}$.  Each rule application will
    increase one statistic, and then the whole sum of the statistics.
\end{definition}

With this new definition we can work with equivalence relations that contain more than one
equivalence.

\begin{theorem}
    Let $R$ be a system composed of the single-image systems $E_i$ which fulfill
    condition \eqref{joininvariance}. We say $\{\rho_i\} = \img(E_i)$. Then
    $\sum\nolimits_{\rho_i}$ is an increasing statistic with respect to each
    $E_i$ and then by summing these statistics together we have an increasing
    statistic with respect to $R$, namely $\sum_{\img(R)}$.  The rewrite
    relation $\to_R$ is hence terminating.
\end{theorem}

We then go through each partition of $\Sym_3$ and look at all the possible
system a partition can generate. We then apply theorem to each of those systems
and weed out those who don't fulfill the theorem's conditions. So for each
partition we are left with a set of systems that are terminating.

This theorem works well, 176 partitions of the 203 have a terminating system
that can be found this way but for the rest we have to result to a little bit of
manual work.

\begin{example}
    Take the equivalence $\{ \{ 213, 132 \} \}$. Its single-image systems,
    ${\{ 213 \patternrule 132 \}}$ and ${\{ 132 \patternrule 213 \}}$
    don't fulfill \eqref{joininvariance}.
    When we apply some symmetries to it, we get $\{ \{ 312, 231 \} \}$ which has the same problem.

    But for $\{ \{ 213, 132 \} \}$ sigma parameterized by a smaller statistic, namely $21$, is an
    increasing statistic. So we say $132$ is a good image, there is exists an
    increasing statistic with 132 as the system's image.
    so $\{ 213 \patternrule 132 \}$ is a terminating system.
\end{example}

It turns out the rest, 27 out of 203 systems, all contain either $\{ \{ 213, 132
\} \}$ or $\{ \{ 312, 231 \} \}$. When we explicitly give a good image for those
systems then we get a terminating system for all 203 equivalence relations.

With these observation we are able to find terminating systems for all set
partitions on $\Sym_3$.

\subsection{Making a terminating system confluent}
First we can reduce the amount we check by the symmetries of the square.
If we show that a system is both terminating and confluent, then all systems
which we can arrive at through mirroring the permutations etc are also
terminating and confluent.

I have not found one algorithm that can make all systems confluent.
I'm using two.

With a system that is terminating, then it's possible to preserve its
termination when making it confluent.
By ensuring that all rules that are added have a positive sigma difference
we can be certain that a system is still confluent. 

The splitter permutation is the shortest cluster in $\Ocal(R)$.

A sigma difference of a rule $\alpha \patternrule \beta$ under an 
statistic would be
\[
    \Delta(f, \alpha \patternrule \beta) = f(\beta) - f(\alpha)
\]

\begin{theorem}
We use it to check if a statistic is increasing for a certain rule.
If $\Delta(f, \alpha \patternrule \beta) > 0$ for all $\alpha \patternrule \beta
\in R$ then $f$ is an increasing statistic under $R$.
\end{theorem}

\lstset{basicstyle=\footnotesize\ttfamily, frame=single, breaklines=true,
mathescape=true}

\begin{minipage}{\linewidth}
\begin{lstlisting}[title={Algorithm 1}]
algo1(sys):
    in: a terminating system sys
        the increasing statistic that proves sys termination f
    out: a terminating and confluent system

    if sys is confluent then
        return sys

    s := choose shortest cluster in O(R)
    rules := { a -> b | a, b are normal forms of s, $\Delta_f$(a -> b) > 0 }
    rule := rule from rules chosen by maximum sigma difference of rule

    if rules = $\emptyset$ then
        return failure

    return algo(sys + rule)
\end{lstlisting}
\end{minipage}

\begin{minipage}{\linewidth}
\begin{lstlisting}[title={Algorithm 2}]
algo2(sys):
    in: a terminating system sys
    out: a terminating and confluent system

    if sys is confluent then
        return sys

    s := choose shortest cluster in O(R)
    a := first application of R on s
    b := last application of R on s

    rule = a -> b
    if last sys = rule then
        return failure

    return algo(sys + rule)
\end{lstlisting}
\end{minipage}

Both algorithms do not manage to make every terminating system confluent.
Algorithm 1 manages 198 sytems. What are left are the systems:

\begin{gather*}
    \{\{123,213\},\{132,312\},\{231,321\}\} \\
    \{\{123,213\},\{132,231\},\{312,321\}\} \\
    \{\{\{123,213,231,321\},\{132,312\}\}\} \\
    \{\{123,132\},\{213,231\},\{312,321\}\} \\
    \{\{123,132\},\{213,312\},\{231,321\}\} 
\end{gather*}

algoritm 2 manages 3 systems and we are left with
\begin{gather*}
    \{\{123,213\},\{132,312\},\{231,321\}\} \\
    \{\{\{123,213,231,321\},\{132,312\}\}\}
\end{gather*}

For these two we can employ a little trick. By reversing each pattern in the
family of equivalences we get the systems
\begin{gather*}
    \{\{312,321\},\{213,231\},\{123,132\}\} \\
    \{\{123,132,312,321\},\{213,231\}\}
\end{gather*}
both of which have already computed systems. We can be sure that they have the
same solution as their reverses, but only reversed. They have the same cluster
generating function as their reverse.

Together they manage all 203 systems.

\section{Application, result and verification}
There are 10 different Wilf-classes for the equivalences of partitions of $S_3$.

\begin{definition}
    We can give a family of equivalences $J$ a size function which is the same
    as the cardinality of the canonical form of the set of equivalences.

    By this definition $| \{ \{ 123, 321 \}, \{ 312, 231, 213 \} \}| = | \{ \{
    123, 321 \}, \{ 312, 231 \}, \{ 312, 213 \} \} | = 3$, for example.
\end{definition}

Let us collate all the results from Appendix!! into a table to show how the
different generating functions are distributed.

The joint distribution is always on the form $\sum_{m \geq 0}m!(x+C(x))^m$ where $C$
is the cluster generating function corresponding to the system.

\begin{center}
\begin{tabular}{c|c|c|c}
    $|J|$ & $C_J(x)$ & count & OEIS seq. \\
    \hline
    0 & $0$ & 1 & A000142 \\
    1 & $-x^3$ & 15 & A212580 \\
    2 & $-2x^3$ & 63 & A212432 \\
    2 & $\frac{-x^3-x^4}{x^2+x+1}$ & 2 & A212581 \\
    3 & $-3x^3$ & 78 & \emph{Not present} \\
    3 & $\frac{-2x^3-2x^4-x^5}{x^2+x+1}$ & 12 & A212433 \\
    4 & $-4x^3$ & 18 & \emph{Not present} \\
    4 & $\frac{-3x^3-3x^4-2x^5}{x^2+x+1}$ & 12 & \emph{Not present}\\
    4 & $\frac{-4x^3-2x^4}{x^2+x+1}$ & 1 & \emph{Not present} \\ 
    5 & $\frac{-5x^3-3x^4-x^5}{x^2+x+1}$ & 1 & \emph{Not present}\\
\end{tabular}
\end{center}

If we parameterize each $C_J$ with it's corresponding $J$ value then we have
three different types of cluster generating functions.

The most visible should be $C'_J(x) = -|J|x^3$.
\begin{center}
\begin{tabular}{c|c|c|c}
    $|J|$ & $C'(x)$ & count & OEIS seq. \\
    \hline
    0 & $0$ & 1 & A000142 \\
    1 & $-x^3$ & 15 & A212580 \\
    2 & $-2x^3$ & 63 & A212432 \\
    3 & $-3x^3$ & 78 & \emph{Not present} \\
    4 & $-4x^3$ & 18 & \emph{Not present} \\
\end{tabular}
\end{center}
This formula accounts for 175 of the 203 equivalence relations.
What seems to be in common with these equivalences is that the domains of the
single image systems are not clustering with each other. (?)

Next we see $C''_J(x)$:
\[
    C''_J(x) = \frac{-|J|x^3-(|J|-1)x^4-(|J|-2)x^5}{x^2+x+1}
\]

The actual instances of $C''_J$:
\begin{center}
\begin{tabular}{c|c|c|c}
    $|J|$ & $C''(x)$ & count & OEIS seq. \\
    \hline
    2 & $\frac{-x^3-x^4}{x^2+x+1}$ & 2 & A212581 \\
    3 & $\frac{-2x^3-2x^4-x^5}{x^2+x+1}$ & 12 & A212433 \\
    4 & $\frac{-3x^3-3x^4-2x^5}{x^2+x+1}$ & 12 & \emph{Not present}\\
\end{tabular}
\end{center}


Finally we have $C'''_J(x)$ which accounts for only 2 equivalence relations.
\[
    C_3(n,x) = \frac{-nx^3-(n-2)x^4-(n-4)x^5}{x^2+x+1}
\]
with it's instances being
\begin{center}
\begin{tabular}{c|c|c|c}
    $|J|$ & $C'''(x)$ & count & OEIS seq. \\
    \hline
    4 & $\frac{-4x^3-2x^4}{x^2+x+1}$ & 1 & \emph{Not present} \\ 
    5 & $\frac{-5x^3-3x^4-x^5}{x^2+x+1}$ & 1 & \emph{Not present}\\
\end{tabular}
\end{center}

So we see that the partitions of $\Sym_3$ are split into three different types
of Wilf-classes, which we can describe by $C'$, $C''$ and $C'''$ and the size
of the partition.

\subsection{Programming}
There were three small programs written for the project: mp for generating the
systems from the equivalence relations, mp for taking the domain of these
systems and calculating the generating functions and finding if their sequences
were found in the OEIS, and finally tester which calculates the first 6 values
of the sequences manually. mp and tester were written in Haskell and mp in Sage Math. 


\section{Problems, questions and remarks}
Are there larger systems, for example from $\Sym_4$, where we cannot apply the
methods used in this paper? The methods here being the good-image algorithm and
then the moving-pattern algorithm?

Would similar formulas to the $C_n$ appear again when we would
look at $\Sym_4$? Can we justify why we get these patterns?

Can we describe clearly $C_{1,2,3}$ from how patterns in the base system cluster together?
Why do the exact rules used to make the system confluent not seem to matter?

Can we derive a way to create a cluster generating function directly from a
given family of equivalences? Can we create a test which tells us given a family
of equivalences which of $C_1$, $C_2$ and $C_3$ will enumerate the equivalence
classes induced by those family of equivalences?
Can this way be generalized to partitions of $\Sym_n$, for every $n$?

Why can we not use our algorithms for systems .. and .. ? And why do they work
for the symmetries? Does there exist an algorithm that makes all systems confluent?

The limitations of Claesson's method lay mainly in the fact that the domain of a
system must be fully reduced for the calculations to work. For instance, we can not use his
method for $\{ \{ 12, 21 \}, \{ 123, 321, 312, 213 \} \}$.

Is there a universial increasing statistic for say all systems defined the
partitions of $\Sym_3$? Can we prove otherwise?

\begin{conjecture}
    If a partition $J$ on $\Sym_n$ has a system that has no non-self-clusters in
    its domain then its cluster generating function is $x^n-|J|x$.
\end{conjecture}

\section*{Acknowledgments}

\bibliographystyle{plain}
\bibliography{mp}

\newgeometry{left=2cm,right=2cm,top=2cm,bottom=2cm}
\begin{appendices}
\section{Terminating and confluent systems}
We categorize the equivalence families/equivalence relations by what their joint
distribition is.
\begin{multicols}{2}
\allowdisplaybreaks
\begin{scriptsize}
$$
\begin{matrix}
\sum_{m \geq 0} m! \left(
x
\right)^m
\ 
\left(0, 0\right)
\\
\left(1, 2, 6, 24, 120, 720, 5040\right)
\texttt{A000142}
\end{matrix}
$$
\begin{align}
\textnormal{the identity equivalence}
\end{align}
$$
\begin{matrix}
\sum_{m \geq 0} m! \left(
-x^{3} + x
\right)^m
\ 
\left(-x^{3}, 0\right)
\\
\left(1, 2, 5, 20, 102, 626, 4458\right)
\texttt{A212580}
\end{matrix}
$$
\begin{align}
\{\{132, 312\}\}
\ 
&
\begin{matrix}
312 \mapsto 132
\end{matrix}
\\
\{\{132, 231\}\}
\ 
&
\begin{matrix}
231 \mapsto 132
\end{matrix}
\\
\{\{231, 321\}\}
\ 
&
\begin{matrix}
231 \mapsto 321
\end{matrix}
\\
\{\{312, 321\}\}
\ 
&
\begin{matrix}
312 \mapsto 321
\end{matrix}
\\
\{\{132, 321\}\}
\ 
&
\begin{matrix}
132 \mapsto 321
\end{matrix}
\\
\{\{213, 321\}\}
\ 
&
\begin{matrix}
213 \mapsto 321
\end{matrix}
\\
\{\{213, 231\}\}
\ 
&
\begin{matrix}
231 \mapsto 213
\end{matrix}
\\
\{\{213, 312\}\}
\ 
&
\begin{matrix}
312 \mapsto 213
\end{matrix}
\\
\{\{123, 213\}\}
\ 
&
\begin{matrix}
213 \mapsto 123
\end{matrix}
\\
\{\{123, 321\}\}
\ 
&
\begin{matrix}
321 \mapsto 123\\2341 \mapsto 4123
\end{matrix}
\\
\{\{123, 231\}\}
\ 
&
\begin{matrix}
231 \mapsto 123
\end{matrix}
\\
\{\{123, 312\}\}
\ 
&
\begin{matrix}
312 \mapsto 123
\end{matrix}
\\
\{\{123, 132\}\}
\ 
&
\begin{matrix}
132 \mapsto 123
\end{matrix}
\end{align}
$$
\begin{matrix}
\sum_{m \geq 0} m! \left(
-2 x^{3} + x
\right)^m
\ 
\left(-2 x^{3}, 0\right)
\\
\left(1, 2, 4, 16, 84, 536, 3912\right)
\texttt{A212432}
\end{matrix}
$$
\begin{align}
\{\{132, 231, 312\}\}
\ 
&
\begin{matrix}
231 \mapsto 132\\312 \mapsto 132\\2431 \mapsto 4132\\35412 \mapsto 45132
\end{matrix}
\\
\{\{132, 312\}, \{231, 321\}\}
\ 
&
\begin{matrix}
132 \mapsto 312\\231 \mapsto 321
\end{matrix}
\\
\{\{132, 231, 321\}\}
\ 
&
\begin{matrix}
132 \mapsto 321\\231 \mapsto 321
\end{matrix}
\\
\{\{132, 231\}, \{312, 321\}\}
\ 
&
\begin{matrix}
312 \mapsto 132\\213 \mapsto 123
\end{matrix}
\\
\{\{132, 312, 321\}\}
\ 
&
\begin{matrix}
132 \mapsto 321\\312 \mapsto 321
\end{matrix}
\\
\{\{132, 312\}, \{213, 321\}\}
\ 
&
\begin{matrix}
312 \mapsto 132\\321 \mapsto 213\\3241 \mapsto 4213\\43512 \mapsto 54132
\end{matrix}
\\
\{\{132, 231\}, \{213, 321\}\}
\ 
&
\begin{matrix}
231 \mapsto 132\\321 \mapsto 213\\3241 \mapsto 4213\\35421 \mapsto 45213
\end{matrix}
\\
\{\{213, 231, 321\}\}
\ 
&
\begin{matrix}
213 \mapsto 321\\231 \mapsto 321
\end{matrix}
\\
\{\{213, 312, 321\}\}
\ 
&
\begin{matrix}
213 \mapsto 321\\312 \mapsto 321
\end{matrix}
\\
\{\{132, 213, 321\}\}
\ 
&
\begin{matrix}
132 \mapsto 321\\213 \mapsto 321\\3214 \mapsto 1432\\32154 \mapsto 21543
\end{matrix}
\\
\{\{132, 312\}, \{213, 231\}\}
\ 
&
\begin{matrix}
312 \mapsto 132\\231 \mapsto 213\\2431 \mapsto 4213\\43512 \mapsto 45132
\end{matrix}
\\
\{\{213, 231\}, \{312, 321\}\}
\ 
&
\begin{matrix}
231 \mapsto 132\\213 \mapsto 123
\end{matrix}
\\
\{\{132, 321\}, \{213, 231\}\}
\ 
&
\begin{matrix}
321 \mapsto 132\\231 \mapsto 213\\2431 \mapsto 4132\\43521 \mapsto 45132
\end{matrix}
\\
\{\{213, 231, 312\}\}
\ 
&
\begin{matrix}
231 \mapsto 213\\312 \mapsto 213\\3241 \mapsto 4213\\43512 \mapsto 45213
\end{matrix}
\\
\{\{132, 213, 231\}\}
\ 
&
\begin{matrix}
132 \mapsto 231\\213 \mapsto 231\\2314 \mapsto 1342\\23154 \mapsto 21453
\end{matrix}
\\
\{\{132, 231\}, \{213, 312\}\}
\ 
&
\begin{matrix}
231 \mapsto 132\\312 \mapsto 213\\3241 \mapsto 4132\\35412 \mapsto 45213
\end{matrix}
\\
\{\{213, 312\}, \{231, 321\}\}
\ 
&
\begin{matrix}
213 \mapsto 312\\231 \mapsto 321
\end{matrix}
\\
\{\{132, 321\}, \{213, 312\}\}
\ 
&
\begin{matrix}
321 \mapsto 132\\312 \mapsto 213\\2431 \mapsto 4132\\35412 \mapsto 54213
\end{matrix}
\\
\{\{132, 213, 312\}\}
\ 
&
\begin{matrix}
132 \mapsto 312\\213 \mapsto 312\\3124 \mapsto 1423\\31254 \mapsto 21534
\end{matrix}
\\
\{\{123, 213\}, \{132, 312\}\}
\ 
&
\begin{matrix}
213 \mapsto 123\\312 \mapsto 132
\end{matrix}
\\
\{\{123, 213\}, \{132, 231\}\}
\ 
&
\begin{matrix}
213 \mapsto 123\\231 \mapsto 132
\end{matrix}
\\
\{\{123, 213\}, \{231, 321\}\}
\ 
&
\begin{matrix}
213 \mapsto 123\\231 \mapsto 321
\end{matrix}
\\
\{\{123, 213\}, \{312, 321\}\}
\ 
&
\begin{matrix}
213 \mapsto 123\\312 \mapsto 321
\end{matrix}
\\
\{\{123, 213\}, \{132, 321\}\}
\ 
&
\begin{matrix}
213 \mapsto 123\\321 \mapsto 132\\2431 \mapsto 4132
\end{matrix}
\\
\{\{123, 213, 321\}\}
\ 
&
\begin{matrix}
213 \mapsto 123\\321 \mapsto 123\\2341 \mapsto 4123
\end{matrix}
\\
\{\{123, 213, 231\}\}
\ 
&
\begin{matrix}
213 \mapsto 123\\231 \mapsto 123
\end{matrix}
\\
\{\{123, 213, 312\}\}
\ 
&
\begin{matrix}
213 \mapsto 123\\312 \mapsto 123
\end{matrix}
\\
\{\{123, 321\}, \{132, 312\}\}
\ 
&
\begin{matrix}
321 \mapsto 123\\312 \mapsto 132\\2341 \mapsto 4123\\34512 \mapsto 54132
\end{matrix}
\\
\{\{123, 321\}, \{132, 231\}\}
\ 
&
\begin{matrix}
321 \mapsto 123\\132 \mapsto 231\\2341 \mapsto 4123
\end{matrix}
\\
\{\{123, 321\}, \{213, 231\}\}
\ 
&
\begin{matrix}
321 \mapsto 123\\231 \mapsto 213\\2341 \mapsto 4123\\43521 \mapsto 45123
\end{matrix}
\\
\{\{123, 321\}, \{213, 312\}\}
\ 
&
\begin{matrix}
123 \mapsto 321\\312 \mapsto 213\\3214 \mapsto 1432
\end{matrix}
\\
\{\{123, 231, 321\}\}
\ 
&
\begin{matrix}
123 \mapsto 321\\231 \mapsto 321\\3214 \mapsto 1432
\end{matrix}
\\
\{\{123, 312, 321\}\}
\ 
&
\begin{matrix}
123 \mapsto 321\\312 \mapsto 321\\3214 \mapsto 1432
\end{matrix}
\\
\{\{123, 132, 321\}\}
\ 
&
\begin{matrix}
132 \mapsto 123\\321 \mapsto 123\\2341 \mapsto 4123
\end{matrix}
\\
\{\{123, 231\}, \{132, 312\}\}
\ 
&
\begin{matrix}
231 \mapsto 123\\312 \mapsto 132\\2431 \mapsto 4123\\34512 \mapsto 45132
\end{matrix}
\\
\{\{123, 231\}, \{312, 321\}\}
\ 
&
\begin{matrix}
321 \mapsto 132\\213 \mapsto 123\\2431 \mapsto 4132
\end{matrix}
\\
\{\{123, 231\}, \{132, 321\}\}
\ 
&
\begin{matrix}
231 \mapsto 123\\321 \mapsto 132\\2431 \mapsto 4132\\34521 \mapsto 45132
\end{matrix}
\\
\{\{123, 231\}, \{213, 321\}\}
\ 
&
\begin{matrix}
231 \mapsto 123\\321 \mapsto 213\\3241 \mapsto 4213\\34521 \mapsto 45213
\end{matrix}
\\
\{\{123, 231\}, \{213, 312\}\}
\ 
&
\begin{matrix}
231 \mapsto 123\\312 \mapsto 213\\3241 \mapsto 4123\\34512 \mapsto 45213
\end{matrix}
\\
\{\{123, 231, 312\}\}
\ 
&
\begin{matrix}
231 \mapsto 123\\312 \mapsto 123\\2341 \mapsto 4123\\34512 \mapsto 45123
\end{matrix}
\\
\{\{123, 132, 231\}\}
\ 
&
\begin{matrix}
132 \mapsto 123\\231 \mapsto 123
\end{matrix}
\\
\{\{123, 312\}, \{132, 231\}\}
\ 
&
\begin{matrix}
312 \mapsto 123\\231 \mapsto 132\\2341 \mapsto 4132\\35412 \mapsto 45123
\end{matrix}
\\
\{\{123, 312\}, \{231, 321\}\}
\ 
&
\begin{matrix}
123 \mapsto 312\\231 \mapsto 321\\3124 \mapsto 1423
\end{matrix}
\\
\{\{123, 312\}, \{132, 321\}\}
\ 
&
\begin{matrix}
312 \mapsto 123\\132 \mapsto 321
\end{matrix}
\\
\{\{123, 312\}, \{213, 321\}\}
\ 
&
\begin{matrix}
312 \mapsto 123\\321 \mapsto 213\\3241 \mapsto 4213\\43512 \mapsto 54123
\end{matrix}
\\
\{\{123, 312\}, \{213, 231\}\}
\ 
&
\begin{matrix}
312 \mapsto 123\\231 \mapsto 213\\2341 \mapsto 4213\\43512 \mapsto 45123
\end{matrix}
\\
\{\{123, 132, 312\}\}
\ 
&
\begin{matrix}
132 \mapsto 123\\312 \mapsto 123
\end{matrix}
\\
\{\{123, 132\}, \{231, 321\}\}
\ 
&
\begin{matrix}
132 \mapsto 123\\231 \mapsto 321
\end{matrix}
\\
\{\{123, 132\}, \{312, 321\}\}
\ 
&
\begin{matrix}
132 \mapsto 123\\312 \mapsto 321
\end{matrix}
\\
\{\{123, 132\}, \{213, 321\}\}
\ 
&
\begin{matrix}
213 \mapsto 123\\321 \mapsto 132\\2431 \mapsto 4132
\end{matrix}
\\
\{\{123, 132\}, \{213, 231\}\}
\ 
&
\begin{matrix}
213 \mapsto 123\\312 \mapsto 132
\end{matrix}
\\
\{\{123, 132\}, \{213, 312\}\}
\ 
&
\begin{matrix}
213 \mapsto 123\\231 \mapsto 132
\end{matrix}
\end{align}
$$
\begin{matrix}
\sum_{m \geq 0} m! \left(
\frac{-x^{4} - x^{3} + x^{2} + x}{x^{2} + x + 1}
\right)^m
\ 
\left(-x^{2} - x + 2, \frac{-x - 2}{x^{2} + x + 1}\right)
\\
\left(1, 2, 4, 17, 89, 556, 4011\right)
\texttt{A212581}
\end{matrix}
$$
\begin{align}
\{\{231, 312, 321\}\}
\ 
&
\begin{matrix}
231 \mapsto 321\\312 \mapsto 321
\end{matrix}
\\
\{\{123, 132, 213\}\}
\ 
&
\begin{matrix}
132 \mapsto 123\\213 \mapsto 123
\end{matrix}
\end{align}
$$
\begin{matrix}
\sum_{m \geq 0} m! \left(
\frac{-x^{5} - 2 x^{4} - 2 x^{3} + x^{2} + x}{x^{2} + x + 1}
\right)^m
\ 
\left(-x^{3} - x^{2} - x + 2, \frac{-x - 2}{x^{2} + x + 1}\right)
\\
\left(1, 2, 3, 13, 71, 470, 3497\right)
\texttt{A212433}
\end{matrix}
$$
\begin{align}
\{\{132, 231, 312, 321\}\}
\ 
&
\begin{matrix}
132 \mapsto 321\\231 \mapsto 321\\312 \mapsto 321
\end{matrix}
\\
\{\{213, 231, 312, 321\}\}
\ 
&
\begin{matrix}
213 \mapsto 321\\231 \mapsto 321\\312 \mapsto 321
\end{matrix}
\\
\{\{123, 213\}, \{231, 312, 321\}\}
\ 
&
\begin{matrix}
213 \mapsto 123\\231 \mapsto 321\\312 \mapsto 321
\end{matrix}
\\
\{\{123, 132, 213, 321\}\}
\ 
&
\begin{matrix}
132 \mapsto 123\\213 \mapsto 123\\321 \mapsto 123\\2341 \mapsto 4123
\end{matrix}
\\
\{\{123, 132, 213, 231\}\}
\ 
&
\begin{matrix}
132 \mapsto 123\\213 \mapsto 123\\231 \mapsto 123
\end{matrix}
\\
\{\{123, 132, 213, 312\}\}
\ 
&
\begin{matrix}
132 \mapsto 123\\213 \mapsto 123\\312 \mapsto 123
\end{matrix}
\\
\{\{123, 132, 213\}, \{231, 321\}\}
\ 
&
\begin{matrix}
132 \mapsto 123\\213 \mapsto 123\\231 \mapsto 321
\end{matrix}
\\
\{\{123, 132, 213\}, \{312, 321\}\}
\ 
&
\begin{matrix}
132 \mapsto 123\\213 \mapsto 123\\312 \mapsto 321
\end{matrix}
\\
\{\{123, 231, 312, 321\}\}
\ 
&
\begin{matrix}
123 \mapsto 321\\231 \mapsto 321\\312 \mapsto 321\\3214 \mapsto 1432
\end{matrix}
\\
\{\{123, 132\}, \{231, 312, 321\}\}
\ 
&
\begin{matrix}
132 \mapsto 123\\231 \mapsto 321\\312 \mapsto 321
\end{matrix}
\end{align}
$$
\begin{matrix}
\sum_{m \geq 0} m! \left(
-3 x^{3} + x
\right)^m
\ 
\left(-3 x^{3}, 0\right)
\\
\left(1, 2, 3, 12, 66, 450, 3402\right)
\texttt{Not found}
\end{matrix}
$$
\begin{align}
\{\{132, 231, 312\}, \{213, 321\}\}
\ 
&
\begin{matrix}
231 \mapsto 132\\312 \mapsto 132\\321 \mapsto 213\\2431 \mapsto 4132\\3241 \mapsto 4213\\35412 \mapsto 45132\\45213 \mapsto 54132\\43512 \mapsto 54132\\465213 \mapsto 546132
\end{matrix}
\\
\{\{132, 312\}, \{213, 231, 321\}\}
\ 
&
\begin{matrix}
132 \mapsto 312\\213 \mapsto 321\\231 \mapsto 321\\3124 \mapsto 1432\\32154 \mapsto 21534
\end{matrix}
\\
\{\{132, 213, 231, 321\}\}
\ 
&
\begin{matrix}
132 \mapsto 321\\213 \mapsto 321\\231 \mapsto 321\\3214 \mapsto 1432\\32154 \mapsto 21543
\end{matrix}
\\
\{\{132, 231\}, \{213, 312, 321\}\}
\ 
&
\begin{matrix}
312 \mapsto 132\\213 \mapsto 123\\231 \mapsto 123\\2431 \mapsto 4123\\34512 \mapsto 45132
\end{matrix}
\\
\{\{132, 213, 312, 321\}\}
\ 
&
\begin{matrix}
132 \mapsto 321\\213 \mapsto 321\\312 \mapsto 321\\3214 \mapsto 1432\\32154 \mapsto 21543
\end{matrix}
\\
\{\{132, 312, 321\}, \{213, 231\}\}
\ 
&
\begin{matrix}
213 \mapsto 321\\231 \mapsto 321\\132 \mapsto 312\\3124 \mapsto 1432\\32154 \mapsto 21534
\end{matrix}
\\
\{\{132, 321\}, \{213, 231, 312\}\}
\ 
&
\begin{matrix}
321 \mapsto 132\\231 \mapsto 213\\312 \mapsto 213\\2431 \mapsto 4132\\3241 \mapsto 4213\\35412 \mapsto 54213\\45132 \mapsto 54213\\43512 \mapsto 45213\\546132 \mapsto 465213
\end{matrix}
\\
\{\{132, 213, 231\}, \{312, 321\}\}
\ 
&
\begin{matrix}
231 \mapsto 132\\312 \mapsto 132\\213 \mapsto 123\\2431 \mapsto 4132\\35412 \mapsto 45132
\end{matrix}
\\
\{\{132, 231, 321\}, \{213, 312\}\}
\ 
&
\begin{matrix}
123 \mapsto 312\\132 \mapsto 312\\213 \mapsto 231\\3124 \mapsto 1423\\1342 \mapsto 1423\\31254 \mapsto 12534\\23145 \mapsto 21534\\23154 \mapsto 21534\\312564 \mapsto 312645\\231564 \mapsto 231645
\end{matrix}
\\
\{\{132, 213, 312\}, \{231, 321\}\}
\ 
&
\begin{matrix}
132 \mapsto 312\\213 \mapsto 312\\231 \mapsto 321\\3124 \mapsto 1423\\31254 \mapsto 21534
\end{matrix}
\\
\{\{123, 213\}, \{132, 231, 312\}\}
\ 
&
\begin{matrix}
213 \mapsto 123\\231 \mapsto 132\\312 \mapsto 132\\2431 \mapsto 4132\\35412 \mapsto 45132
\end{matrix}
\\
\{\{123, 213\}, \{132, 231, 321\}\}
\ 
&
\begin{matrix}
231 \mapsto 321\\123 \mapsto 312\\132 \mapsto 312\\3124 \mapsto 1423\\31254 \mapsto 12534
\end{matrix}
\\
\{\{123, 213\}, \{132, 312, 321\}\}
\ 
&
\begin{matrix}
213 \mapsto 123\\312 \mapsto 132\\321 \mapsto 132\\2431 \mapsto 4132\\35412 \mapsto 54132
\end{matrix}
\\
\{\{123, 213, 321\}, \{132, 312\}\}
\ 
&
\begin{matrix}
213 \mapsto 123\\321 \mapsto 123\\312 \mapsto 132\\2341 \mapsto 4123\\34512 \mapsto 54132
\end{matrix}
\\
\{\{123, 213, 321\}, \{132, 231\}\}
\ 
&
\begin{matrix}
213 \mapsto 123\\321 \mapsto 123\\231 \mapsto 132\\2341 \mapsto 4123\\35421 \mapsto 45123
\end{matrix}
\\
\{\{123, 213, 231, 321\}\}
\ 
&
\begin{matrix}
123 \mapsto 321\\132 \mapsto 321\\231 \mapsto 321\\3214 \mapsto 1432\\32154 \mapsto 12543
\end{matrix}
\\
\{\{123, 213, 312, 321\}\}
\ 
&
\begin{matrix}
213 \mapsto 123\\312 \mapsto 123\\321 \mapsto 123\\2341 \mapsto 4123\\34512 \mapsto 54123
\end{matrix}
\\
\{\{123, 213, 231\}, \{132, 312\}\}
\ 
&
\begin{matrix}
213 \mapsto 123\\231 \mapsto 123\\312 \mapsto 132\\2431 \mapsto 4123\\34512 \mapsto 45132
\end{matrix}
\\
\{\{123, 213, 231\}, \{312, 321\}\}
\ 
&
\begin{matrix}
123 \mapsto 312\\132 \mapsto 312\\231 \mapsto 321\\3124 \mapsto 1423\\31254 \mapsto 12534
\end{matrix}
\\
\{\{123, 213, 231\}, \{132, 321\}\}
\ 
&
\begin{matrix}
213 \mapsto 123\\231 \mapsto 123\\321 \mapsto 132\\2431 \mapsto 4132\\34521 \mapsto 45132
\end{matrix}
\\
\{\{123, 213, 231, 312\}\}
\ 
&
\begin{matrix}
213 \mapsto 123\\231 \mapsto 123\\312 \mapsto 123\\2341 \mapsto 4123\\34512 \mapsto 45123
\end{matrix}
\\
\{\{123, 213, 312\}, \{132, 231\}\}
\ 
&
\begin{matrix}
213 \mapsto 321\\231 \mapsto 321\\132 \mapsto 312\\3124 \mapsto 1432\\32154 \mapsto 21534
\end{matrix}
\\
\{\{123, 213, 312\}, \{231, 321\}\}
\ 
&
\begin{matrix}
312 \mapsto 132\\321 \mapsto 132\\213 \mapsto 123\\2431 \mapsto 4132\\35412 \mapsto 54132
\end{matrix}
\\
\{\{123, 213, 312\}, \{132, 321\}\}
\ 
&
\begin{matrix}
213 \mapsto 123\\312 \mapsto 123\\321 \mapsto 132\\2431 \mapsto 4132\\35412 \mapsto 54123
\end{matrix}
\\
\{\{123, 321\}, \{132, 231, 312\}\}
\ 
&
\begin{matrix}
321 \mapsto 123\\231 \mapsto 132\\312 \mapsto 132\\2341 \mapsto 4123\\2431 \mapsto 4132\\34512 \mapsto 54132\\45123 \mapsto 54132\\35412 \mapsto 45132\\465123 \mapsto 456132
\end{matrix}
\\
\{\{123, 321\}, \{132, 312\}, \{213, 231\}\}
\ 
&
\begin{matrix}
321 \mapsto 123\\312 \mapsto 132\\231 \mapsto 213\\2341 \mapsto 4123\\2431 \mapsto 4213\\34512 \mapsto 54132\\43521 \mapsto 45123\\43512 \mapsto 45132
\end{matrix}
\\
\{\{123, 321\}, \{213, 231, 312\}\}
\ 
&
\begin{matrix}
321 \mapsto 123\\231 \mapsto 213\\312 \mapsto 213\\2341 \mapsto 4123\\3241 \mapsto 4213\\34512 \mapsto 54213\\45123 \mapsto 54213\\43512 \mapsto 45213\\546123 \mapsto 456213
\end{matrix}
\\
\{\{123, 321\}, \{132, 213, 231\}\}
\ 
&
\begin{matrix}
123 \mapsto 321\\132 \mapsto 231\\213 \mapsto 231\\3214 \mapsto 1432\\2314 \mapsto 1342\\32154 \mapsto 12453\\21543 \mapsto 12453\\23154 \mapsto 21453\\231654 \mapsto 321564
\end{matrix}
\\
\{\{123, 321\}, \{132, 231\}, \{213, 312\}\}
\ 
&
\begin{matrix}
321 \mapsto 123\\231 \mapsto 132\\312 \mapsto 213\\2341 \mapsto 4123\\3241 \mapsto 4132\\34512 \mapsto 54213\\35421 \mapsto 45123\\35412 \mapsto 45213
\end{matrix}
\\
\{\{123, 321\}, \{132, 213, 312\}\}
\ 
&
\begin{matrix}
123 \mapsto 321\\132 \mapsto 312\\213 \mapsto 312\\3214 \mapsto 1432\\3124 \mapsto 1423\\32154 \mapsto 12534\\21543 \mapsto 12534\\31254 \mapsto 21534\\312654 \mapsto 321645
\end{matrix}
\\
\{\{123, 231, 321\}, \{132, 312\}\}
\ 
&
\begin{matrix}
123 \mapsto 321\\231 \mapsto 321\\132 \mapsto 312\\3214 \mapsto 1432\\32154 \mapsto 12534
\end{matrix}
\\
\{\{123, 231, 321\}, \{213, 312\}\}
\ 
&
\begin{matrix}
123 \mapsto 321\\231 \mapsto 321\\213 \mapsto 312\\3214 \mapsto 1432\\31245 \mapsto 21543
\end{matrix}
\\
\{\{123, 132, 231, 321\}\}
\ 
&
\begin{matrix}
123 \mapsto 321\\132 \mapsto 321\\231 \mapsto 321\\3214 \mapsto 1432\\32154 \mapsto 12543
\end{matrix}
\\
\{\{123, 312, 321\}, \{132, 231\}\}
\ 
&
\begin{matrix}
213 \mapsto 123\\321 \mapsto 123\\312 \mapsto 132\\2341 \mapsto 4123\\34512 \mapsto 54132
\end{matrix}
\\
\{\{123, 312, 321\}, \{213, 231\}\}
\ 
&
\begin{matrix}
213 \mapsto 123\\321 \mapsto 123\\231 \mapsto 132\\2341 \mapsto 4123\\35421 \mapsto 45123
\end{matrix}
\\
\{\{123, 132, 312, 321\}\}
\ 
&
\begin{matrix}
132 \mapsto 123\\312 \mapsto 123\\321 \mapsto 123\\2341 \mapsto 4123\\34512 \mapsto 54123
\end{matrix}
\\
\{\{123, 132, 321\}, \{213, 231\}\}
\ 
&
\begin{matrix}
123 \mapsto 321\\132 \mapsto 321\\213 \mapsto 231\\3214 \mapsto 1432\\1342 \mapsto 1432\\32154 \mapsto 12543\\23145 \mapsto 21543\\23154 \mapsto 21543\\321564 \mapsto 321654\\231564 \mapsto 231654
\end{matrix}
\\
\{\{123, 132, 321\}, \{213, 312\}\}
\ 
&
\begin{matrix}
123 \mapsto 321\\132 \mapsto 321\\213 \mapsto 312\\3214 \mapsto 1432\\1423 \mapsto 1432\\32154 \mapsto 12543\\31245 \mapsto 21543\\31254 \mapsto 21543\\321645 \mapsto 321654\\312645 \mapsto 312654
\end{matrix}
\\
\{\{123, 231\}, \{132, 312, 321\}\}
\ 
&
\begin{matrix}
321 \mapsto 132\\213 \mapsto 123\\312 \mapsto 123\\2431 \mapsto 4132\\35412 \mapsto 54123
\end{matrix}
\\
\{\{123, 231\}, \{132, 312\}, \{213, 321\}\}
\ 
&
\begin{matrix}
231 \mapsto 123\\312 \mapsto 132\\321 \mapsto 213\\2431 \mapsto 4123\\3241 \mapsto 4213\\34512 \mapsto 45132\\34521 \mapsto 45213\\43512 \mapsto 54132
\end{matrix}
\\
\{\{123, 231\}, \{213, 312, 321\}\}
\ 
&
\begin{matrix}
321 \mapsto 132\\213 \mapsto 123\\231 \mapsto 123\\2431 \mapsto 4132\\34521 \mapsto 45132
\end{matrix}
\\
\{\{123, 231\}, \{132, 213, 321\}\}
\ 
&
\begin{matrix}
123 \mapsto 231\\132 \mapsto 321\\213 \mapsto 321\\2314 \mapsto 1342\\3214 \mapsto 1432\\23154 \mapsto 12543\\21453 \mapsto 12543\\32154 \mapsto 21543\\321564 \mapsto 231654
\end{matrix}
\\
\{\{123, 231\}, \{132, 321\}, \{213, 312\}\}
\ 
&
\begin{matrix}
231 \mapsto 123\\321 \mapsto 132\\312 \mapsto 213\\2431 \mapsto 4132\\3241 \mapsto 4123\\34521 \mapsto 45132\\34512 \mapsto 45213\\35412 \mapsto 54213
\end{matrix}
\\
\{\{123, 231\}, \{132, 213, 312\}\}
\ 
&
\begin{matrix}
123 \mapsto 231\\132 \mapsto 312\\213 \mapsto 312\\2314 \mapsto 1342\\3124 \mapsto 1423\\23154 \mapsto 12534\\21453 \mapsto 12534\\31254 \mapsto 21534\\312564 \mapsto 231645
\end{matrix}
\\
\{\{123, 231, 312\}, \{132, 321\}\}
\ 
&
\begin{matrix}
231 \mapsto 123\\312 \mapsto 123\\321 \mapsto 132\\2341 \mapsto 4123\\2431 \mapsto 4132\\34512 \mapsto 45123\\45132 \mapsto 54123\\35412 \mapsto 54123\\456132 \mapsto 465123
\end{matrix}
\\
\{\{123, 231, 312\}, \{213, 321\}\}
\ 
&
\begin{matrix}
231 \mapsto 123\\312 \mapsto 123\\321 \mapsto 213\\2341 \mapsto 4123\\3241 \mapsto 4213\\34512 \mapsto 45123\\45213 \mapsto 54123\\43512 \mapsto 54123\\456213 \mapsto 546123
\end{matrix}
\\
\{\{123, 132, 231, 312\}\}
\ 
&
\begin{matrix}
132 \mapsto 123\\231 \mapsto 123\\312 \mapsto 123\\2341 \mapsto 4123\\34512 \mapsto 45123
\end{matrix}
\\
\{\{123, 132, 231\}, \{312, 321\}\}
\ 
&
\begin{matrix}
312 \mapsto 132\\321 \mapsto 132\\213 \mapsto 123\\2431 \mapsto 4132\\35412 \mapsto 54132
\end{matrix}
\\
\{\{123, 132, 231\}, \{213, 321\}\}
\ 
&
\begin{matrix}
123 \mapsto 231\\132 \mapsto 231\\213 \mapsto 321\\2314 \mapsto 1342\\1432 \mapsto 1342\\23154 \mapsto 12453\\32145 \mapsto 21453\\32154 \mapsto 21453\\231654 \mapsto 231564\\321654 \mapsto 321564
\end{matrix}
\\
\{\{123, 132, 231\}, \{213, 312\}\}
\ 
&
\begin{matrix}
123 \mapsto 231\\132 \mapsto 231\\213 \mapsto 312\\2314 \mapsto 1342\\1423 \mapsto 1342\\23154 \mapsto 12453\\31245 \mapsto 21453\\31254 \mapsto 21453\\231645 \mapsto 231564\\312645 \mapsto 312564
\end{matrix}
\\
\{\{123, 312\}, \{132, 231, 321\}\}
\ 
&
\begin{matrix}
123 \mapsto 312\\132 \mapsto 321\\231 \mapsto 321\\3124 \mapsto 1423\\31254 \mapsto 12543
\end{matrix}
\\
\{\{123, 312\}, \{132, 231\}, \{213, 321\}\}
\ 
&
\begin{matrix}
312 \mapsto 123\\231 \mapsto 132\\321 \mapsto 213\\2341 \mapsto 4132\\3241 \mapsto 4213\\35412 \mapsto 45123\\35421 \mapsto 45213\\43512 \mapsto 54123
\end{matrix}
\\
\{\{123, 312\}, \{213, 231, 321\}\}
\ 
&
\begin{matrix}
123 \mapsto 312\\213 \mapsto 321\\231 \mapsto 321\\3124 \mapsto 1423\\32145 \mapsto 21534
\end{matrix}
\\
\{\{123, 312\}, \{132, 213, 321\}\}
\ 
&
\begin{matrix}
123 \mapsto 312\\132 \mapsto 321\\213 \mapsto 321\\3124 \mapsto 1423\\3214 \mapsto 1432\\31254 \mapsto 12543\\21534 \mapsto 12543\\32154 \mapsto 21543\\321645 \mapsto 312654
\end{matrix}
\\
\{\{123, 312\}, \{132, 321\}, \{213, 231\}\}
\ 
&
\begin{matrix}
312 \mapsto 123\\321 \mapsto 132\\231 \mapsto 213\\2341 \mapsto 4213\\2431 \mapsto 4132\\35412 \mapsto 54123\\43512 \mapsto 45123\\43521 \mapsto 45132
\end{matrix}
\\
\{\{123, 312\}, \{132, 213, 231\}\}
\ 
&
\begin{matrix}
123 \mapsto 312\\132 \mapsto 231\\213 \mapsto 231\\3124 \mapsto 1423\\2314 \mapsto 1342\\31254 \mapsto 12453\\21534 \mapsto 12453\\23154 \mapsto 21453\\231645 \mapsto 312564
\end{matrix}
\\
\{\{123, 132, 312\}, \{231, 321\}\}
\ 
&
\begin{matrix}
123 \mapsto 312\\132 \mapsto 312\\231 \mapsto 321\\3124 \mapsto 1423\\31254 \mapsto 12534
\end{matrix}
\\
\{\{123, 132, 312\}, \{213, 321\}\}
\ 
&
\begin{matrix}
123 \mapsto 312\\132 \mapsto 312\\213 \mapsto 321\\3124 \mapsto 1423\\1432 \mapsto 1423\\31254 \mapsto 12534\\32145 \mapsto 21534\\32154 \mapsto 21534\\312654 \mapsto 312645\\321654 \mapsto 321645
\end{matrix}
\\
\{\{123, 132, 312\}, \{213, 231\}\}
\ 
&
\begin{matrix}
123 \mapsto 312\\132 \mapsto 312\\213 \mapsto 231\\3124 \mapsto 1423\\1342 \mapsto 1423\\31254 \mapsto 12534\\23145 \mapsto 21534\\23154 \mapsto 21534\\312564 \mapsto 312645\\231564 \mapsto 231645
\end{matrix}
\\
\{\{123, 132\}, \{213, 231, 321\}\}
\ 
&
\begin{matrix}
213 \mapsto 123\\312 \mapsto 132\\321 \mapsto 132\\2431 \mapsto 4132\\35412 \mapsto 54132
\end{matrix}
\\
\{\{123, 132\}, \{213, 312, 321\}\}
\ 
&
\begin{matrix}
231 \mapsto 321\\123 \mapsto 312\\132 \mapsto 312\\3124 \mapsto 1423\\31254 \mapsto 12534
\end{matrix}
\\
\{\{123, 132\}, \{213, 231, 312\}\}
\ 
&
\begin{matrix}
213 \mapsto 123\\231 \mapsto 132\\312 \mapsto 132\\2431 \mapsto 4132\\35412 \mapsto 45132
\end{matrix}
\end{align}
$$
\begin{matrix}
\sum_{m \geq 0} m! \left(
\frac{-2 x^{5} - 3 x^{4} - 3 x^{3} + x^{2} + x}{x^{2} + x + 1}
\right)^m
\ 
\left(-2 x^{3} - x^{2} - x + 2, \frac{-x - 2}{x^{2} + x + 1}\right)
\\
\left(1, 2, 2, 9, 53, 388, 3019\right)
\texttt{Not found}
\end{matrix}
$$
\begin{align}
\{\{132, 213, 231, 312, 321\}\}
\ 
&
\begin{matrix}
132 \mapsto 321\\213 \mapsto 321\\231 \mapsto 321\\312 \mapsto 321\\3214 \mapsto 1432\\32154 \mapsto 21543
\end{matrix}
\\
\{\{123, 213, 231, 312, 321\}\}
\ 
&
\begin{matrix}
123 \mapsto 321\\132 \mapsto 321\\231 \mapsto 321\\312 \mapsto 321\\3214 \mapsto 1432\\32154 \mapsto 12543
\end{matrix}
\\
\{\{123, 132, 213, 231, 321\}\}
\ 
&
\begin{matrix}
123 \mapsto 321\\132 \mapsto 321\\231 \mapsto 321\\312 \mapsto 321\\3214 \mapsto 1432\\32154 \mapsto 12543
\end{matrix}
\\
\{\{123, 132, 213, 312, 321\}\}
\ 
&
\begin{matrix}
132 \mapsto 123\\213 \mapsto 123\\312 \mapsto 123\\321 \mapsto 123\\2341 \mapsto 4123\\34512 \mapsto 54123
\end{matrix}
\\
\{\{123, 132, 213, 231, 312\}\}
\ 
&
\begin{matrix}
132 \mapsto 123\\213 \mapsto 123\\231 \mapsto 123\\312 \mapsto 123\\2341 \mapsto 4123\\34512 \mapsto 45123
\end{matrix}
\\
\{\{123, 132, 231, 312, 321\}\}
\ 
&
\begin{matrix}
123 \mapsto 321\\132 \mapsto 321\\231 \mapsto 321\\312 \mapsto 321\\3214 \mapsto 1432\\32154 \mapsto 12543
\end{matrix}
\end{align}
$$
\begin{matrix}
\sum_{m \geq 0} m! \left(
-4 x^{3} + x
\right)^m
\ 
\left(-4 x^{3}, 0\right)
\\
\left(1, 2, 2, 8, 48, 368, 2928\right)
\texttt{Not found}
\end{matrix}
$$
\begin{align}
\{\{123, 213, 231, 321\}, \{132, 312\}\}
\ 
&
\begin{matrix}
123 \mapsto 321\\132 \mapsto 321\\231 \mapsto 321\\213 \mapsto 312\\3214 \mapsto 1432\\1423 \mapsto 1432\\32154 \mapsto 12543\\31245 \mapsto 21543\\31254 \mapsto 21543\\321645 \mapsto 321654\\312645 \mapsto 312654
\end{matrix}
\\
\{\{123, 213, 312, 321\}, \{132, 231\}\}
\ 
&
\begin{matrix}
213 \mapsto 123\\312 \mapsto 123\\321 \mapsto 123\\231 \mapsto 132\\4132 \mapsto 2341\\2341 \mapsto 4123\\34512 \mapsto 54123\\35412 \mapsto 45123\\35421 \mapsto 45123\\456132 \mapsto 456123\\465132 \mapsto 465123
\end{matrix}
\\
\{\{123, 213, 231, 312\}, \{132, 321\}\}
\ 
&
\begin{matrix}
213 \mapsto 123\\231 \mapsto 123\\312 \mapsto 123\\321 \mapsto 132\\2341 \mapsto 4123\\2431 \mapsto 4132\\34512 \mapsto 45123\\45132 \mapsto 54123\\35412 \mapsto 54123\\456132 \mapsto 465123
\end{matrix}
\\
\{\{123, 132, 231, 321\}, \{213, 312\}\}
\ 
&
\begin{matrix}
123 \mapsto 321\\132 \mapsto 321\\231 \mapsto 321\\213 \mapsto 312\\3214 \mapsto 1432\\1423 \mapsto 1432\\32154 \mapsto 12543\\31245 \mapsto 21543\\31254 \mapsto 21543\\321645 \mapsto 321654\\312645 \mapsto 312654
\end{matrix}
\\
\{\{123, 132, 312, 321\}, \{213, 231\}\}
\ 
&
\begin{matrix}
213 \mapsto 123\\312 \mapsto 123\\321 \mapsto 123\\231 \mapsto 132\\4132 \mapsto 2341\\2341 \mapsto 4123\\34512 \mapsto 54123\\35412 \mapsto 45123\\35421 \mapsto 45123\\456132 \mapsto 456123\\465132 \mapsto 465123
\end{matrix}
\\
\{\{123, 231\}, \{132, 213, 312, 321\}\}
\ 
&
\begin{matrix}
321 \mapsto 132\\213 \mapsto 123\\231 \mapsto 123\\312 \mapsto 123\\2431 \mapsto 4132\\2341 \mapsto 4123\\35412 \mapsto 54123\\45132 \mapsto 54123\\34512 \mapsto 45123\\456132 \mapsto 465123
\end{matrix}
\\
\{\{123, 132, 231, 312\}, \{213, 321\}\}
\ 
&
\begin{matrix}
213 \mapsto 123\\231 \mapsto 123\\312 \mapsto 123\\321 \mapsto 132\\2341 \mapsto 4123\\2431 \mapsto 4132\\34512 \mapsto 45123\\45132 \mapsto 54123\\35412 \mapsto 54123\\456132 \mapsto 465123
\end{matrix}
\\
\{\{123, 312\}, \{132, 213, 231, 321\}\}
\ 
&
\begin{matrix}
123 \mapsto 312\\132 \mapsto 321\\213 \mapsto 321\\231 \mapsto 321\\3124 \mapsto 1423\\3214 \mapsto 1432\\31254 \mapsto 12543\\21534 \mapsto 12543\\32154 \mapsto 21543\\321645 \mapsto 312654
\end{matrix}
\end{align}
$$
\begin{matrix}
\sum_{m \geq 0} m! \left(
\frac{-2 x^{4} - 3 x^{3} + x^{2} + x}{x^{2} + x + 1}
\right)^m
\ 
\left(-2 x^{2} - 2 x + 4, \frac{-2 x - 4}{x^{2} + x + 1}\right)
\\
\left(1, 2, 2, 10, 58, 408, 3110\right)
\texttt{Not found}
\end{matrix}
$$
\begin{align}
\{\{123, 132, 213\}, \{231, 312, 321\}\}
\ 
&
\begin{matrix}
132 \mapsto 123\\213 \mapsto 123\\231 \mapsto 321\\312 \mapsto 321
\end{matrix}
\end{align}
\end{scriptsize}

\end{multicols}
\end{appendices}

\end{document}
